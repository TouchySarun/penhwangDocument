\chapter{\ifcpe บทนำ\else Introduction\fi}

\section{\ifcpe ที่มาของโครงงาน\else Project rationale\fi}
บริษัทต่าง ๆ ที่จ่ายเงินเดือนตามเวลาทำงานของพนักงาน (นับจากเข้างาน จนถึงออกงาน) ก็จะมีวิธีการเช็คชื่อเข้า-ออกงานของพนักงานที่ต่างกัน 
หลากหลายรูปแบบ เช่น การเซ็นชื่อลงบนกระดาษ การตอกบัตร หรือ การสแกนลายนิ้วมือ
แต่ในปัจจุบันมีบริษัทส่วนหนึ่งได้ตระหนักถึงปัญหาจากการใช้วิธีการเช็คชื่อเข้าออกงานแบบดังกล่าว
เช่น คำนวณเงินเดือนยากเพราะต้องทำการค้นหาข้อมูลจากเอกสารจำนวนมาก 
พนักงานทุจริตด้วยการตอกบัตรแทนกัน 
หรือ พนักงานตอกบัตรผิดใบ 
ซึ่งการใช้มนุษย์ในการบันทึกหรือจัดการข้อมูล มักจะเกิดความผิดพลาดที่เกิดจากมนุษย์ (human error) 
ส่งผลให้เกิดความล่าช้า โดยสังเกตได้จากในวันออกเงินเดือน พนักงานในบางบริษัทจะได้กลับบ้านช้ากว่าปกติเพราะต้องรอคำนวณเงินเดือนให้เสร็จเสียก่อน

จึงมีบริษัทส่วนหนึ่งเลือกที่จะใช้แอปพลิเคชันต่าง ๆ เพื่อที่จะจัดการแก้ไขปัญหาดังกล่าวข้างต้น 
เพราะ สามารถเรียกดูข้อมูลได้ตลอดเวลา 
สรุปผลและคำนวณออกมาเป็นเงินเดือนได้อย่างรวดเร็ว 
สามารถป้องกันการทุจริตของพนักงาน 
รวมไปถึงจัดการการเดินเรื่องขอเอกสารให้มีความสะดวกรวดเร็วมากยิ่งขึ้น และ ลดปัญหาความผิดพลาดที่เกิดขึ้นจากการเก็บข้อมูลโดยใช้มนุษย์ไปพร้อมกัน

แต่แอปพลิเคชันเหล่านี้ก็ยังมีข้อเสีย เช่น 
พนักงานต้องทำการดาวน์โหลดแอปพลิเคชันไว้ในเครื่องส่วนตัวซึ่งจากการสำรวจพนักงานส่วนใหญ่ไม่เต็มใจที่จะดาวน์โหลดแอป ฯ โดยบางแอปพลิเคชันก็ไม่ได้อำนวยความสะดวกในการใช้งานด้านต่าง ๆ เช่น
ไม่มีการแจ้งเตือนเมื่อพนักงานจะต้องเข้าทำงาน กำลังจะเข้างานสาย มีการเปลี่ยนแปลงเวลาการทำงานของตนเอง หรือ คำขอต่าง ๆ ของตนเองถูกยืนยัน/ปฎิเสธ
ซึ่งเป็นสิ่งที่แอปพลิเคชันควรจะรองรับ 
และการเช็คชื่อเข้าทำงานของพนักงานยังทำได้ช้า และ มีหลายขั้นตอนทำให้เวลาที่บันทึกอยู่ในระบบและเวลาที่พนักงานเข้างานจริงต่างกันพอสมควร

ทางผู้พัฒนาเล็งเห็นปัญหาข้างต้นจึงได้พัฒนาโปรเจคนี้ขึ้นโดยการใช้ LINE chatbot มาพัฒนาต่อยอด
เพื่อให้สามารถทำงาน ครอบคลุมฟังก์ชันต่าง ๆ ตามที่แอปพลิเคชันเหล่านั้นทำได้
โดยรักษาข้อดีต่าง ๆ เอาไว้พร้อมกับแก้ปัญหาที่เกิดขึ้นจากการใช้แอปพลิเคชันเหล่านั้นด้วย

\section{\ifcpe วัตถุประสงค์ของโครงงาน\else Objectives\fi}
\begin{enumerate}
    \item พัฒนาแชทบอทที่สามารถทำฟังก์ชันหลัก ๆ ของแอปพลิเคชันเช็คชื่อพนักงานที่มีตามท้องตลาดได้ ประกอบด้วย 
    \begin{enumerate}
    \item[1.1] จัดตารางเข้า-ออกให้กับพนักงาน
    \item[1.2] เช็คชื่อเข้า-ออกบันทึกสถานที่และเวลา
    \item[1.3] การยื่นคำขอต่าง ๆ ของพนักงาน เช่น เปลี่ยนเวลาการทำงานของตนเอง ขอลา
    \item[1.4] ตั้งค่าบริษัทเช่น การเพิ่ม-ลดพนักงาน กะ แผนก สถานที่ที่จะอนุญาติให้พนักงานเช็คชื่อ และ ประเภทคำขอ
    \end{enumerate} 
    \item มีการแจ้งเตือนเมื่อพนักงานจะต้องเข้าทำงาน, กำลังจะเข้างานสาย, มีการเปลี่ยนแปลงเวลาการทำงานของตนเอง หรือ คำขอต่าง ๆ ของตนเองถูกยืนยัน/ปฎิเสธ
\end{enumerate}

\section{\ifcpe ขอบเขตของโครงงาน\else Project scope\fi}

\subsection{\ifcpe ขอบเขตด้านฮาร์ดแวร์\else Hardware scope\fi}
\begin{itemize}
    \item Android version 4.4 เป็นต้นไป (อุปกรณ์ที่รองรับแอปพลิเคชัน LINE)
\end{itemize}

\subsection{\ifcpe ขอบเขตด้านซอฟต์แวร์\else Software scope\fi}
\begin{itemize}
    \item ระบบบันทึกเวลาการเข้าออกงานของพนักงาน
    \item ระบบจัดการคำขอต่าง ๆ ของพนักงาน ประกอบด้วย ขอลา ขอเปลี่ยนกะ และ ขอเข้าร่วมบริษัท
    \item ระบบจัดการตารางเวลาทำงาน 
    \item ระบบจัดการประเภทการลา
    \item ระบบจัดการสถานที่เข้าออกงานของพนักงาน
    \item ระบบจัดการแบ่งกลุ่มพนักงานเป็นแผนก
    \item ระบบแจ้งเตือนเวลาเข้างาน
    \item ระบบแจ้งเตือนเมื่อมีการเปลี่ยนแปลงตารางเวลาการทำงาน
    \item ระบบสรุปประวัติการทำงานของพนักงาน เพื่อให้ง่ายต่อการคำนวณเงินเดือน
\end{itemize}

\section{\ifcpe ประโยชน์ที่ได้รับ\else Expected outcomes\fi}
\begin{itemize}
    \item ช่วยอำนวยความสะดวกในการบันทึกเวลาเข้าออกงานของพนักงาน
    \item ง่ายต่อการตรวจสอบข้อมูลการเข้าออก และ การลางานของพนักงาน
    \item ใช้เวลาในการจัดการคำขอต่าง ๆ น้อยลง
    \item สามารถดูสิทธิ์การลาคงเหลือ และสรุปประวัติการบันทึกเวลาได้ง่ายและทันที
    \item หัวหน้างานสามารถติดตามการเข้าออกงานของลูกน้องได้แบบ real time
    \item หัวหน้างานสามารถจัดการหรือเตรียมตัวก่อนล่วงหน้าเมื่อมีพนักงานขอลา
\end{itemize}
\section{\ifcpe เทคโนโลยีและเครื่องมือที่ใช้\else Technology and tools\fi}
\begin{itemize}
    \item VS code: เป็น code editor
    \item LINE application
    \item LINE API Messenging
    \item LINE Bot Designer​ \end{englang}
    \item Google dialogflow
    \item Google firebase (firestore)
    \item Nuxt.js
\end{itemize}
\section{\ifcpe แผนการดำเนินงาน\else Project plan\fi}
\begin{plan}{7}{2020}{3}{2021}
    \planitem{7}{2020}{9}{2020}{ศึกษาแอปฯปัจจุบัน}
    \planitem{8}{2020}{9}{2020}{ศึกษาการสร้างแชทบอท​}
    \planitem{7}{2020}{9}{2020}{ออกแบบระบบ}
    \planitem{10}{2020}{10}{2020}{ออกแบบแชทบอท​}
    \planitem{7}{2020}{10}{2020}{ออกแบบเว็บ​}
    \planitem{7}{2020}{10}{2020}{ออกแบบฐานข้อมูล}
    \planitem{11}{2020}{12}{2020}{สร้างแชทบอท}
    \planitem{11}{2020}{1}{2021}{เชื่อมแชทบอทกับฐานข้อมูล}
    \planitem{11}{2020}{1}{2021}{สร้างเว็บ}
    \planitem{11}{2020}{2}{2021}{เชื่อมเว็บกับแชทบอท}
    \planitem{11}{2020}{3}{2021}{ทดสอบและแก้ไข Bugs}
\end{plan}

\section{\ifcpe บทบาทและความรับผิดชอบ\else Roles and responsibilities\fi}
\begin{itemize}
    \item น.ส.ธนันพร ยานะ
    \begin{itemize}
        \item web designer​ \end{englang}
        \item chatbot designer​​ \end{englang}
        \item frontend developer​ \end{englang}
    \end{itemize}
    \item นายศรัณญ์ ซือสุวรรณ
    \begin{itemize} 
        \item database admin
        \item full stack developer​​ \end{englang}
    \end{itemize}
\end{itemize}

\section{\ifcpe%
ผลกระทบด้านสังคม สุขภาพ ความปลอดภัย กฎหมาย และวัฒนธรรม
\else%
Impacts of this project on society, health, safety, legal, and cultural issues
\fi}
การนำเป็นห่วง (แชทบอทสำหรับจัดการเวลาทำงานของพนักงาน) มาใช้จะทำให้การจัดการการทำงานพนักงาน ง่าย รวดเร็ว และเพิ่มประสิทธิภาพในการทำงาน ทำให้การเกิดความผิดพลาดที่เกิดจากมนุษย์น้อยลง เนื่องจากใช้ระบบคอมพิวเตอร์ในการจัดการ จึงมีความแม่นยำ และ ถูกต้องโดยข้อมูลเหล่านี้จะถูกบันทึกไว้ในระบบคอมพิวเตอร์ทำให้สามารถเรียกใช้ได้ตามต้องการและมีความปลอดภัยเพิ่มขึ้น สิ่งเหล่านี้อาจทำให้การปฎิบัติตัวของพนักงานเปลี่ยนไป กล่าวคืออาจตรงต่อเวลามากขึ้น และ มีการทำงานแบบเป็นระบบมากขึ้น ตลอดจนทำให้วัฒนธรรมขององค์กรนั้นๆ เปลี่ยนไปในทางที่ดีขึ้นด้วย