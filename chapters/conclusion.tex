\chapter{\ifcpe บทสรุปและข้อเสนอแนะ\else Conclusions and Discussion\fi}

\section{\ifcpe สรุปผล\else Conclusions\fi}
\quad “เป็นห่วง” เป็นระบบที่พัฒนาขึ้นเพื่อเป็นทางเลือกในการจัดการเวลาทํางานของพนักงานสําหรับบริษัทที่ต้องการใช้เทคโนโลยีใหม่ ๆ โดยพนักงานสามารถเข้าถึงได้ง่าย ผ่านไลน์แชทบอท 
ดังนั้นจึงไม่ต้องติดตั้งแอปพลิเคชันเพิ่มเติม ระบบออกแบบให้ช่วยลดความผิดพลาดของมนุษย์ จึงง่ายต่อการจัดการ และ ตรวจสอบ ความถูกต้องโดยฝ่ายบุคคล
โดย เป็นห่วง ได้มีการทดสอบการใช้งานเบื้องต้นแล้ว พบว่ามีประโยชน์ต่อผู้ใช้ และ ง่ายต่อการใช้งาน ทั้งนี้ในอนาคตจะต้องมีการทดสอบ 
และสํารวจผลเพื่อตรวจสอบความถูกต้อง และ ความแม่นยําของระบบเพิ่มเติม เนื่องจากสถานการณ์ COVID-19 ในปัจจุบันทําให้ไม่สามารถทดสอบกับผู้ใช้งานทุกบทบาทได้

\section{\ifcpe ปัญหาที่พบและแนวทางการแก้ไข\else Challenges\fi}

% ในการทำโครงงาน เป็นห่วง ไลน์แชทบอทสำหรับจัดการเวลาทำงานของพนักงาน พบว่าเกิดปัญหาหลัก ๆ ดังนี้
% \begin{itemize}
%   \item ข้อมูลมีความสัมพันธ์กันซับซ้อนไม่เหมาะกับการเก็บข้อมูลแบบ NoSQL
%   \item การเชื่อมต่อกันหลายขั้นหลายระบบทำให้การตอบสนองกับผู้ใช้งานช้าลง
%   \item LINE ยังไม่มีรูปแบบที่รองรับสำหรับการทำงานที่ออกแบบไว้ทั้งหมด
%   \item การพัฒนาระบบได้ช้า เนื่องจากไม่ได้ศึกษาเทคโนโลยีที่จะใช้แบบลึกซึ้งก่อนทำงานจริง
% \end{itemize}
ในการทำโครงงาน เป็นห่วง ไลน์แชทบอทสำหรับจัดการเวลาทำงานของพนักงาน พบว่าเกิดปัญหาที่สำคัญประกอบด้วย 
ข้อมูลของพนักงาน บริษัท และ คำขอต่าง ๆ มีความสัมพันธ์กันซับซ้อนไม่เหมาะกับการเก็บข้อมูลแบบ NoSQL 
แต่สาเหตุที่ผู้พัฒนาเลือกใช้ฐานข้อมูลแบบ NoSQL เพราะในช่วงการออกแบบไม่ได้คำนึกถึงการใช้งานทั้งหมด และ firebase 
มีข้อดีคือสามารถ deploy ฐานข้อมูล และ API ที่เขียนขึ้นมาได้ทันที และ ฐานข้อมูลแบบ NoSQL มีความยืดหยุ่นสามารถรองรับความผิดพลาดจากการใส่ข้อมูลได้ 
ปัญหาต่อมาคือ การเชื่อมต่อกันหลายขั้นหลายระบบทำให้การตอบสนองกับผู้ใช้งานช้าลง 
อีกปัญหาคือ แอปพลิเคชัน LINE ยังไม่มีรูปแบบที่รองรับสำหรับการทำงานที่ออกแบบไว้ทั้งหมด 
ปัญหาที่สำคัญสุดท้ายคือ การพัฒนาระบบได้ช้า เนื่องจากไม่ได้ศึกษาเทคโนโลยีที่จะใช้แบบลึกซึ้งก่อนทำงานจริง

\section{\ifcpe%
ข้อเสนอแนะและแนวทางการพัฒนาต่อ
\else%
Suggestions and further improvements
\fi
}

% ข้อเสนอแนะเพื่อพัฒนาโครงงานนี้ต่อไป หรือ สำหรับผู้พัฒนาทีมอื่น ๆ ที่จะทำโครงงานที่คล้ายกันในอนาคต มีดังนี้
% \begin{itemize}
%   \item ในขั้นตอนการออกแบบระบบ ให้คำนึงถึงการนำข้อมูลไปใช้งานด้วย โดยหากการดึงข้อมูลแต่ละครั้งมีความจำเป็นต้องใช้ข้อมูลจากหลายแหล่งพร้อม ๆ กันควรเป็นการเก็บแบบ SQL มากกว่าเพื่อความสะดวกในการจัดการข้อมูล
%   \item หากเลือกที่จะใช้แบบ NoSQL ก็ควรศึกษาประโยชน์ และ ข้อเสีย รวมถึงข้อควรระวังในการออกแบบฐานข้อมูลแบบนี้ด้วย
%   \item นอกจากศึกษาเทคโนโลยีที่จะใช้แล้ว ควรมีการลองนำเทคโนโลยีเหล่านั้นมาใช้งานกับโปรเจคเล็ก ๆ ก่อน เพื่อจะเปลี่ยนแปลง แก้ไขได้ทันที
%   \item หากจะให้ "การใช้งานง่าย" เป็นจุดเด่นของโปรเจค ควรมีการทำ prototype ไปทดสอบกับผู้ใช้งานบ่อย ๆ เพื่อนำข้อคิดเห็นมาปรับปรุง
% \end{itemize}


ข้อเสนอแนะเพื่อพัฒนาโครงงานนี้ต่อไป หรือ สำหรับผู้พัฒนาทีมอื่น ๆ ที่จะทำโครงงานที่คล้ายกันในอนาคต ประกอบด้วย 
ในขั้นตอนการออกแบบระบบ ผู้พัฒนาควรคำนึงถึงการนำข้อมูลไปใช้งานทั้งหมด หมายความว่าต้องออกแบบระบบ และ ฟังก์ชันทั้งหมดก่อนจะเลือกใช้ฐานข้อมูล 
โดยหากการดึงข้อมูลไปใช้งานแต่ละครั้งมีความจำเป็นต้องใช้ข้อมูลจากหลายจุดพร้อม ๆ กันควรเป็นการเก็บแบบ SQL มากกว่าเพื่อความสะดวกในการจัดการข้อมูล 
แต่หากเลือกที่จะใช้แบบ NoSQL ก็ควรศึกษาประโยชน์ และ ข้อเสีย รวมถึงข้อควรระวังในการออกแบบฐานข้อมูลแบบนี้ด้วย 
ต่อไปคือ นอกจากศึกษาเทคโนโลยีที่จะใช้แล้ว ควรมีการลองนำเทคโนโลยีเหล่านั้นมาใช้งานกับโครงงานเล็ก ๆ ก่อน เพื่อให้รู้ผลจากการเพิ่มเทคโนโลยีนี้เข้าไปในระบบ และ เพื่อให้รู้วิธีการใช้งานที่ถูกต้อง 
คำแนะนำข้อสุดท้ายคือ หากผู้พัฒนาต้องการให้ "การใช้งานง่าย" เป็นจุดเด่นของโครงการควรมีการทำ prototype ไปทดสอบกับผู้ใช้งานบ่อย ๆ เพื่อนำข้อคิดเห็นมาปรับปรุง จะส่งผลให้ผลสุดท้ายของโครงการออกมาตรงตามความต้องการของผู้ใช้งานมากกว่าตอนนี้
