\chapter{\ifcpe ทฤษฎีที่เกี่ยวข้อง\else Background Knowledge and Theory\fi}

\quad การทำโครงงาน เริ่มต้นด้วยการศึกษาค้นคว้า ทฤษฎีที่เกี่ยวข้อง หรือ งานวิจัย/โครงงาน ที่เคยมีผู้นําเสนอไว้แล้ว
ซึ่งเนื้อหาในบทนี้ก็จะเกี่ยวกับการอธิบายถึงสิ่งที่เกี่ยวข้องกับโครงงาน 
เพื่อให้ผู้อ่านเข้าใจเนื้อหาในบทถัด ๆ ไปได้ง่ายขึ้น เนื้อหาในบทนี้จะแบ่งออกเป็นสี่ส่วนหลัก ๆ คือ 
แอปพลิเคชันในตลาดปัจจุบัน, LINE และ ส่วนเสริมของ LINE, เทคโนโลยีที่เกี่ยวข้องอื่น ๆ ดังนี้ 
\section{แอปพลิเคชันในตลาดปัจจุบัน}

\subsection{PAYDAY}
\quad SME payday ระบบ HRM (human resources management) สำหรับผู้ประกอบการ SMEs และ พนักงาน 
โดยสามารถจัดการงานเอกสาร และ เข้าถึงงานบุคคลได้สะดวก ทุกที่ทุกเวลา ไม่ว่าจะเป็นการยืนยันเวลาเข้าออกในการทำงาน 
การยื่นคำร้องต่าง ๆ รวมถึงการเบิกค่าใช้จ่ายไปจนถึงการอนุมัติพร้อมทั้งฟังก์ชันใช้งานเฉพาะบุคคลสำหรับการเช็กข้อมูลส่วนตัว 
ทั้งในเรื่องของเงินเดือนรวมถึงภาษี ประกันสังคม เพื่อให้ทุกกระบวนการง่าย สะดวก รวดเร็วยิ่งขึ้น ใช้งานได้หลายรูปแบบทั้งบน website และ mobile application 
\cite{payday}
\begin{itemize}
  \item จุดเด่นของแอปพลิเคชัน
  \begin{itemize}
    \item มีระบบคำนวณและจ่ายเงินเดือนอัตโนมัติ
    \item สามารถสร้างnไฟล์การจ่ายเงินเดือนสำหรับธนาคารไทยพาณิชย์
  \end{itemize}
  \item ความคิดเห็นจากผู้ใช้งานใน App store และ Play store
  \begin{itemize}
    \item ต้องมีการเข้าสู่ระบบก่อนจะเช็คชื่อทุกครั้ง ทำให้ใช้เวลานาน
    \item ระบบเข้าใช้งานไม่เสถียร กรอกข้อมูลเหมือนเดิมทุกอย่างแต่ได้ผลลัพธ์ไม่เหมือนกัน
  \end{itemize}
\end{itemize}

\subsection{JOBCAN}
\quad ระบบจัดการการเข้างานผ่าน cloud system ที่มีบริษัทเลือกใช้งานมากกว่า 10,000 แห่งทั่วโลก 
มีวิธีตอกบัตรเข้างานหลากหลาย เช่น IC card, โทรศัพท์มือถือผ่าน GPS, สแกนลายนิ้วมือ สามารถสร้างผลัดงาน และ 
รับผลัดงานที่พนักงานต้องการผ่านหน้าจอได้โดยตรง รวมถึงตรวจสอบสถานการณ์ทำงานของพนักงานได้แบบ real-time
\cite{jobcan}
\begin{itemize}
  \item จุดเด่นของแอปพลิเคชัน
  \begin{itemize}
    \item มีระบบคำนวณและจ่ายเงินเดือนอัตโนมัติ
    \item สามารถสร้างnไฟล์การจ่ายเงินเดือนสำหรับธนาคารไทยพาณิชย์
  \end{itemize}
  \item ความคิดเห็นจากผู้ใช้งานใน App store และ Play store
  \begin{itemize}
    \item ต้องมีการเข้าสู่ระบบก่อนจะเช็คชื่อทุกครั้ง ทำให้ใช้เวลานาน
    \item ระบบเข้าใช้งานไม่เสถียร กรอกข้อมูลเหมือนเดิมทุกอย่างแต่ได้ผลลัพธ์ไม่เหมือนกัน
  \end{itemize}
\end{itemize}

\subsection{we-la-dee}
\quad เวลาดีทางเลือกใหม่ของระบบบันทึกเวลาทำงานสำหรับ องค์กรที่ทันสมัย 
ใช้สำหรับการบันทึกเวลาเข้าออกงานที่บริษัทของคุณเพื่อรับข้อมูลเวลาการเข้า-ออกงานที่ถูกต้องแบบ Real-time 
พนักงานสามารถบันทึกเวลาเข้า-ออกงานได้ทั้งที่สำนักงานใหญ่ สำนักงานสาขา หรือพื้นที่ทำงานนอกสถานที่ 
ผู้บริหารสามารถตรวจสอบเวลาการเข้า-ออกงานของพนักงาน ชั่วโมงการทำงานได้ทันทีผ่านโทรศัพท์มือถือ 
สามารถดูรายงาน สถิติการทำงาน ได้ทุกที่ทุกเวลา ช่วยให้การบริหารจัดการระบบเวลาการทำงานในองค์กรเป็นเรื่องง่าย 
และพนักงานเองก็สามารถตรวจสอบชั่วโมงการทำงานผ่านโทรศัพท์มือถือของตนเองได้เช่นกัน 
\cite{weladee}
\begin{itemize}
  \item จุดเด่นของแอปพลิเคชัน
  \begin{itemize}
    \item สามารถบันทึกข้อมูลโดยใช้ RFID ได้ 
    \item สามารถแปลงข้อมูลเป็นไฟล์ PDF ได้ 
    \item พนักงานสามารถลงทะเบียนลาพักร้อนได้ โดยระบบจะส่งข้อมูลไปยังแผนก HR ให้อัตโนมัติ
  \end{itemize}
  \item ความคิดเห็นจากผู้ใช้งานใน App store และ Play store
  \begin{itemize}
    \item สะดวกดีครับ ไม่ต้องรอคิวให้เสียเวลา 
    \item แอปพลิเคชันมีการอัปเดตบ่อย 
    \item การสแกนเข้างานยาก บางครั้งใช้เวลานาน  
  \end{itemize}
\end{itemize}

\subsection{OneDee}
\quad วันดี คือ ระบบบริหารจัดการทีมงานผ่าน chatbot ด้วยหน้าจอที่ใช้งานง่าย ช่วยให้การบริหารงาน HR มีประสิทธิภาพมากขึ้น 
พนักงานสามารถลงเวลาทำงาน ตอกบัตร ขาดลา ด้วยการส่งข้อความผ่าน chatbot และ สามารถพูดคุยกันภายในแอพผ่านหน้าจอแชทช่วยให้ทำงานง่ายยิ่งขึ้น 
อีกทั้งยังสามารถให้บริการเวลาทำงานของพนักงานผ่าน AI ลงเวลาตอกบัตร ลงเวลาพนักงาน เพิ่มประสิทธิภาพการบริหารงานด้วย AI ร่วมกับระบบ HR และ chatbot 
ที่สามารถตอบสนอกงการทำงานภายในองค์การ และ ทำงานร่วมกับพนักงานได้อย่างง่ายดาย ทีมงานสามารถพูดคุยสื่อสารภายในองค์กร และขอแบบฟอร์มต่าง ๆ ได้อย่างรวดเร็ว  
\cite{onedee}
\begin{itemize}
  \item จุดเด่นของแอปพลิเคชัน
  \begin{itemize}
    \item ระบบ UI เป็นมิตรสามารถใช้งานได้ง่าย 
    \item สามารถเข้า-ออกงานได้หลายวิธีทั้ง WIFI, GPS, QR, iBeacon 
  \end{itemize}
  \item ความคิดเห็นจากผู้ใช้งานใน App store และ Play store
  \begin{itemize}
    \item หาตำแหน่งเข้า-ออกงานยาก (GPS)  
    \item Log in ยาก ไม่เสถียร  
    \item บางครั้งเกิดปัญหาใช้ QR code เข้า-ออกงานแล้วแอป ฯ catch 
    \item ลงเวลาพร้อมกันหลายคนแล้วมีปัญหา 
  \end{itemize}
\end{itemize}

\subsection{TimeMint}
\quad แอปพลิเคชันบันทึกเข้า-ออกงานทั้งใน และ นอกสถานที่ เพื่อใช้ทดแทนเครื่องตอกบัตร และ เอกสารใบลาทุกประเภท 
เพื่อให้หัวหน้าและพนักงานใช้งาน และ จัดการทั้งเรื่องขาด ลา มาสาย การขออนุมัติปรับปรุงเวลา 
และการบันทึกเวลาเข้า-ออกงานมีการพัฒนาเพื่อให้ครอบคลุมการใช้งานในการจัดการพนักงาน 
เพื่อให้ใช้งานได้อย่างกว้าวงขวาง โดยไม่จำเป็นต้องใช้เครื่องอื่น เพียงการใช้งานผ่านโทรศัพท์มือถือเท่านั้นก็เพียงพอต่อการใช้งานระบบแล้ว  
\cite{timemint}
\begin{itemize}
  \item จุดเด่นของแอปพลิเคชัน
  \begin{itemize}
    \item สามารถเข้า-ออกงานได้หลายวิธีทั้ง WIFI, GPS, QR, iBeacon 
    \item ค่าบริการยืดหยุ่น 
    \item สามารถบันทึกข้อมูลเข้า-ออกงานผ่านทาง LINE ได้ 
  \end{itemize}
  \item ความคิดเห็นจากผู้ใช้งานใน App store และ Play store
  \begin{itemize}
    \item หาตำแหน่งเข้า-ออกงานช้า (GPS) กว่าจะหาเจอก็เข้างานสายแล้ว  
    \item Ux ไม่ friendly กับผู้ใช้ไม่รู้สถานะว่าเข้า-ออกงานอยู่ แอปกินทรัพยากรมาก 
  \end{itemize}
\end{itemize}

\subsection{HumanOs}
\quad เว็บแอปพลิเคชันที่จัดการการเข้า-ออกงานของพนักงานและระบบเงินเดือน 
โดยไม่ต้องใช้เครื่องมือใด ๆ สามารถจัดการวันลาผ่านระบบออนไลน์ 
จัดการคำนวณเงินเดือนรวมถึงระบบภาษี และทุกอย่างยังสามารถ port เป็น excel ได้  
\cite{humanos}
\begin{itemize}
  \item จุดเด่นของแอปพลิเคชัน
  \begin{itemize}
    \item สามารถแปลงข้อมูลเป็นไฟล์ PDF หรือ excel ได้ 
    \item อนุมัติงานต่าง ๆ ผ่านแอปได้ 
  \end{itemize}
  \item ความคิดเห็นจากผู้ใช้งานใน App store และ Play store
  \begin{itemize}
    \item การลาทำได้ยาก 
    \item มีปัญหาการเชื่อมต่อ 
    \item ฟรีกับธุรกิจขนาดเล็ก (ผู้ใช้งานไม่เกิน 10 คน) 
    \item ไม่รองรับบน iOS 
    \item ใช้ง่ายกว่าแสกนนิ้ว 
  \end{itemize}
\end{itemize}

\section{LINE application}
\quad LINE คือ แอปพลิเคชันที่ผสมผสานบริการ Messaging และ Voice Over IP นำมาผนวกเข้าด้วยกัน จึงทำให้เกิดเป็นแอปพลิชันที่สามารถแชท สร้างกลุ่ม ส่งข้อความ โพสต์รูปต่าง ๆ  หรือจะโทรคุยกันแบบเสียงก็ได้  โดยข้อมูลทั้งหมดไม่ต้องเสียเงิน หากเราใช้งานโทรศัพท์ที่มีแพคเกจอินเทอร์เน็ตอยู่แล้ว แถมยังสามารถใช้งานร่วมกันระหว่าง iOS และ Android รวมทั้งระบบปฏิบัติการอื่น ๆ ได้อีกด้วย การทำงานของ LINE นั้น มีลักษณะคล้าย ๆ กับ WhatsApp ที่ต้องใช้เบอร์โทรศัพท์เพื่อยืนยันการใช้งาน แต่ LINE ได้เพิ่มลูกเล่นอื่น ๆ เข้ามา ทำให้ LINE มีจุดเด่นที่เหนือกว่า WhatsApp  

\subsection{จุดเด่นของ LINE} 
\subsubsection{การสนทนาด้วยเสียงฟรี (Free voice calls)} 
\quad ผู้ใช้งานสามารถโทรหาผู้ที่ใช้ LINE ด้วยกันได้ โดยใช้งานผ่านเครือข่าย 3G และ Wi-Fi เพื่อส่งข้อมูลรูปแบบเสียง โดยไม่มีค่าใช้จ่ายใด ๆ  
\subsubsection{ส่งข้อความแบบวิดีโอและเสียง (Send videos \& voice message)} 
\quad นอกจากการแชทด้วยการส่งข้อความแบบปกติแล้ว LINE ยังสามารถอัดภาพวิดีโอหรือเสียงแล้วส่งไปให้เพื่อน ๆ ได้อีกด้วย โดยสามารถส่งได้เป็นคลิปวิดีโอหรือเสียงในแบบสั้น ๆ ความยาวไม่กี่วินาที   
\subsubsection{สติกเกอร์ (Stickers \& Emoticons)}
\quad อีกหนึ่งความสนุกของแชททั่วไปที่ขาดไม่ได้ก็คืออีโมติคอนน่ารัก ๆ ที่ช่วยเพิ่มสีสันให้การแชทสนุกสนานยิ่งขึ้น และสำหรับ LINE มีทั้ง Stickers และ Emoticons รูปแบบต่าง ๆ และยังเลือกดาวน์โหลดเพิ่มเติมได้อีกด้วย ทำให้ผู้ใช้งานหลายคนติดอกติดใจกับ Stickers และ Emoticons น่ารัก ๆ ของ LINE  
\subsubsection{ปรับแต่งภาพพื้นหลัง (Customizable Wallpaper)}
\quad สามารถเปลี่ยน Wallpaper ในหน้าต่างแชทได้ โดยจะมีภาพ Wallpaper มาให้ทั้งหมด 23 แบบ และสามารถเพิ่ม Wallpaper ที่ต้องการ โดยนำรูปที่อยู่ในโทรศัพท์มือถือมาใช้งานเป็น Wallpaper ได้  
\subsubsection{การสนทนาแบบกลุ่ม (Group chat)} 
\quad LINE สามารถสร้างกลุ่มเพื่อพูดคุยกันได้ หากต้องการความเป็นส่วนตัว อยากคุยเฉพาะกลุ่ม LINE ก็สามารถสร้างกลุ่มเอาไว้พูดคุยได้ 
\subsubsection{Timeline} 
\quad LINE มีความเป็นโซเชียลเน็ตเวิร์กในตัว มี Timeline ให้สามารถอัปเดตสถานะ โพสต์รูป คอมเมนต์ หรือกดถูกใจได้เหมือนกับ Facebook เลยทีเดียว 
\subsubsection{การเพิ่มเพื่อน (Add friends / Contacts)}
\quad LINE สามารถเพิ่ม Contacts จากรายชื่อในโทรศัพท์หากมีเพื่อนคนไหนใช้แอปพลิเคชันนี้อยู่ จะมีสัญลักษณ์ LINE แสดงให้เห็นและสามารถเพิ่มเป็นเพื่อนได้ทันที QR Code สามารถสแกน QR Code ของเพื่อนเพื่อเพิ่มเป็นเพื่อนใน LINE และสามารถสร้าง QR Code ของเราเอง เพื่อใช้สำหรับให้เพื่อน ๆ คนอื่น มาสแกน QR Code เพื่อเพิ่มเพื่อนใน LINE ได้ Shake it! เขย่าโทรศัพท์มือถือ เป็นวิธีการเพิ่มเพื่อนที่เจ๋งสุด ๆ ของ LINE ใช้ในกรณีที่ทั้งสองโทรศัพท์สองเครื่องอยู่ด้วยกัน เมื่อเขย่าเครื่องพร้อม ๆ กัน ก็สามารถเพิ่มเป็นเพื่อนกันได้ Search by ID คือ เราสามารถค้นหาเพื่อนได้จาก ID (คล้าย ๆ กับ PIN ของ BB) โดยการพิมพ์ ID ของเพื่อนที่ต้องการ 

\subsection{LINE Messaging API} 
\quad Line Messaging API คือ การสื่อสารระหว่างบริการของคุณและผู้ใช้ LINE เป็นการสื่อสารแบบสองฝ่าย จะทำให้คุณสามารถให้บริการได้ในห้องแชท LINE เพื่อการให้บริการที่เหมาะสมสำหรับผู้ใช้ LINE แต่ละคนและ Messaging API จะส่งและรับข้อมูลระหว่าง Server ของคุณและแอพ LINE ผ่านทาง Server ของทาง LINE การส่งคำขอจะใช้ API แบบ JSON Messaging API ทำการเชื่อมต่อระหว่าง User ผ่านทาง LINE official account ซึ่ง Messaging API จะสามารถตอบรับเพื่อนรวมถึงส่งข้อความหา User คนอื่น ๆ ที่ Add account เราเป็นเพื่อนโดยผ่านหน้า LINE Manager ที่เราตั้งไว้หรือส่งออกจากจาก Server ของเราก็ได้ในรูปแบบ interactive โต้ตอบ  การใช้งาน Messaging API ทำให้คุณสามารถส่งข้อมูลระหว่าง Server ของเรา ไปยัง User LINE ผ่านทาง LINE Platform ซึ่ง Request ที่ใช้ส่งข้อมูลต้องอยู่ในรูป JSON format โดยตัว Server เราจะต้องเชื่อมต่อกับ LINE Platform และเมื่อ มี User เพิ่ม Account LINE เราเป็นเพื่อน หรือ ส่งข่อความมาหาเรา ทาง LINE Platform จะทำการส่ง Request มายัง Server ที่เราลงทะเบียนผูกไว้กับ LINE account นั้นทันที วิธีนี้เรียกว่า Webhook ซึ่งมันทำให้ผู้ใช้งานรู้สึกเหมือนกับว่าได้โต้ตอบกับคนจริง ๆ 
\cite{lineAPI}
\subsection{LINE Bot} 
\quad LINE Bot คือ Line Official Account ที่ได้นำ Messaging API มาใช้ เป็นบริการ API ตัวหนึ่งที่เปิดให้บริการสำหรับนักพัฒนา โดยเจ้าของ Line Official Account จะทำการกำหนดหรือตั้งค่าไว้ด้านหลังบ้านของบริการ เพื่อให้สามารถโต้ตอบกับผู้ใช้งานได้โดยที่ไม่ต้องใช้คนมาเป็นคนตอบ ซึ่งนี่คือข้อดีของการใช้บริการตอนนี้ เพราะนอกจากจะทำให้ผู้ใช้ใช้งานได้ง่ายมากขึ้นแล้ว ผู้ที่เป็นแอดมินก็จะสะดวกสบายมากขึ้นเช่นกัน เพราะไม่ต้องมาคอยตอบคำถามที่ถามซ้ำ ๆ หรือไม่จำเป็นต้องมานั่งเก็บข้อมูลทีละคน ช่วยให้ผู้ใช้งานแก้ไขปัญหาได้ในเบื้องต้นอย่างว่องไว ไม่ต้องรอคอยเป็นเวลานาน สร้างความประทับใจ ปิดการขายได้เร็วขึ้นและลดต้นทุนในการจ้างแอดมินเพื่อมาคอยตอบคำถามตลอดเวลา เพราะบริการนี้จะช่วยเหลือคุณได้ทุกอย่างที่สามารถทำได้ 
\subsubsection{การสร้าง LINE Bot โดยใช้ Dialogflow} 
\quad ประกอบด้วย 2 ส่วน คือ LINE Official Account เป็นส่วนที่เราต้องสร้างขึ้นเพื่อใช้ในการทำ LINE Bot ที่ไว้ใช้ในการโต้ตอบกับ Dialogflow สามารถกำหนดได้ทั้ง สติกเกอร์ รูปภาพ ข้อความ และ วีดีโอ และ Dialogflow เป็นแพลตฟอร์มที่สามารถช่วยในการพัฒนา LINE Bot ได้สามารถแบ่งได้ 2 กรณีดังนี้ 1.การเขียน Dialogflow ขั้นพื้นฐานไม่จำเป็นต้องทำการเขียนโปรแกรมเลย เนื่องจากเราสามารถพิมพ์ข้อความต่าง ๆที่ใช้สำหรับการถาม-ตอบได้เลย 2.การเขียน Dialogflow ขั้นสูงอาจจะมีการเขียนโปรแกรมเพื่อเพิ่มความสามารถของ LINE Bot ได้ เช่น การส่ง Location การส่งรูปภาพ การส่งสติกเกอร์ เป็นต้น 
\cite{lineBot} \cite{lineDf} \cite{lineDfPyFb}
\section{เทคโนโลยีที่เกี่ยวข้องอื่น ๆ}

\subsubsection{Visual Studio Code (VS Code)}
\quad Visual Studio Code หรือ VS Code เป็นโปรแกรม Code Editor ที่ใช้ในการแก้ไขและปรับแต่งโค้ด โดยมาจากค่ายไมโครซอฟท์ ที่มีการพัฒนาออกมาในรูปแบบของ Opensource จึงสามารถนำมาใช้งานได้แบบฟรี ๆ ทีสต้องการความเป็นมืออาชีพ ซึ่ง Visual Studio Code นั้น เหมาะสำหรับนักพัฒนาโปรแกรมที่ต้องการใช้งานกับแพลตฟอร์ม มีการรองรับการใช้งานทั้งบน Windows, macOS และ Linux มีการสนับสนุนทั้งภาษา  JavaScript, TypeScript และ Node.js สามารถเชื่อมต่อกับ  Git ได้ สามารถนำมาใช้งานได้ง่ายไม่ซับซ้อน มีเครื่องมือส่วนขยายต่าง ๆ ให้เราเลือกใช้อย่างมาก ไม่ว่าจะเป็น 1.การเปิดใช้งานภาษาอื่น ๆ ทั้ง ภาษา  C++,  C\#, Java, Python, PHP หรือ Go  2.Themes 3.Debugger 4.Commands 
\cite{vscode}
\subsubsection{Google Dialogflow} 
\quad Dialogflow คือ Platform สำหรับสร้าง chatbot ของ Google ที่ใช้ machine learning ด้าน Natural Language Processing (NLP) มาช่วยในทำความเข้าใจถึงความต้องการ (Intent) และสิ่งที่ต้องการ (Entity) ในประโยคสนทนาของผู้ใช้งานและตอบคำถามตามความต้องการของผู้ใช้งาน ตามกฎที่ผู้พัฒนาวางเอาไว้ ซึ่ง Dialogflow จะช่วยเพิ่มความยืดหยุ่นของประโยคที่ Chatbot รับมา ว่าไม่จำเป็นต้องตรงตามเงื่อนไขก็สามารถเข้าใจถึงความต้องการของผู้ใช้งานได้ 
\cite{lineDf}
\subsubsection{Google Firebase} 
\quad Firebase คือ Platform ที่รวบรวมเครื่องมือต่าง ๆ สำหรับการจัดการในส่วนของ Backend (Server side) ซึ่งทำให้สามารถ Build Mobile Application ได้อย่างมีประสิทธิภาพ และยังลดเวลาและค่าใช้จ่ายของการทำ Server side หรือการวิเคราะห์ข้อมูลให้อีกด้วย โดยมีทั้งเครื่องมือที่ฟรี และเครื่องมีที่มีค่าใช้จ่าย (สำหรับการต่อขยาย) 
\cite{firebase}
\subsubsection{Nuxt.js }
\quad Nuxt.js คือ Framework ที่นำ Vue.js มาสร้าง web application เสริมความสามารถในการทำ SSR และ Progressive Web Application (PWA) 
\cite{nuxt}

% \section{Third section}
% Section 3 text. The dielectric constant\index{dielectric constant}
% at the air-metal interface determines
% the resonance shift\index{resonance shift} as absorption or capture occurs
% is shown in Equation~\eqref{eq:dielectric}:

% \begin{equation}\label{eq:dielectric}
% k_1=\frac{\omega}{c({1/\varepsilon_m + 1/\varepsilon_i})^{1/2}}=k_2=\frac{\omega
% \sin(\theta)\varepsilon_\mathit{air}^{1/2}}{c}
% \end{equation}

% \noindent
% where $\omega$ is the frequency of the plasmon, $c$ is the speed of
% light, $\varepsilon_m$ is the dielectric constant of the metal,
% $\varepsilon_i$ is the dielectric constant of neighboring insulator,
% and $\varepsilon_\mathit{air}$ is the dielectric constant of air.

% \section{About using figures in your report}

% % define a command that produces some filler text, the lorem ipsum.
% \newcommand{\loremipsum}{
%   \textit{Lorem ipsum dolor sit amet, consectetur adipisicing elit, sed do
%   eiusmod tempor incididunt ut labore et dolore magna aliqua. Ut enim ad
%   minim veniam, quis nostrud exercitation ullamco laboris nisi ut
%   aliquip ex ea commodo consequat. Duis aute irure dolor in
%   reprehenderit in voluptate velit esse cillum dolore eu fugiat nulla
%   pariatur. Excepteur sint occaecat cupidatat non proident, sunt in
%   culpa qui officia deserunt mollit anim id est laborum.}\par}

% \begin{figure}
%   \centering

%   \fbox{
%      \parbox{.6\textwidth}{\loremipsum}
%   }

%   % To include an image in the figure, say myimage.pdf, you could use
%   % the following code. Look up the documentation for the package
%   % graphicx for more information.
%   % \includegraphics[width=\textwidth]{myimage}

%   \caption[Sample figure]{This figure is a sample containing \gls{lorem ipsum},
%   showing you how you can include figures and glossary in your report.
%   You can specify a shorter caption that will appear in the List of Figures.}
%   \label{fig:sample-figure}
% \end{figure}

% \loremipsum\loremipsum

% % This code demonstrates how to get a landscape table or figure. It
% % uses the package lscape to turn everything but the page number into
% % landscape orientation. Everything should be included within an
% % \afterpage{ .... } to avoid causing a page break too early.
% \afterpage{
%   \begin{landscape}
%   \begin{table}
%     \caption{Sample landscape table}
%     \label{tab:sample-table}

%     \centering

%     \begin{tabular}{c||c|c}
%         Year & A & B \\
%         \hline\hline
%         1989 & 12 & 23 \\
%         1990 & 4 & 9 \\
%         1991 & 3 & 6 \\
%     \end{tabular}
%   \end{table}
%   \end{landscape}
% }

% \loremipsum\loremipsum\loremipsum

% \section{Overfull hbox}

% When the \verb.semifinal. option is passed to the \verb.cpecmu. document class,
% any line that is longer than the line width, i.e., an overfull hbox, will be
% highlighted with a black solid rule:
% \begin{center}
% \begin{minipage}{2em}
% juxtaposition
% \end{minipage}
% \end{center}

% \section{\ifcpe%
% ความรู้ตามหลักสูตรซึ่งถูกนำมาใช้หรือบูรณาการในโครงงาน
% \else%
% ISNE knowledge used, applied, or integrated in this project
% \fi
% }

% อธิบายถึงความรู้ และแนวทางการนำความรู้ต่างๆ ที่ได้เรียนตามหลักสูตร ซึ่งถูกนำมาใช้ในโครงงาน

% \section{\ifcpe%
% ความรู้นอกหลักสูตรซึ่งถูกนำมาใช้หรือบูรณาการในโครงงาน
% \else%
% Extracurricular knowledge used, applied, or integrated in this project
% \fi
% }

% อธิบายถึงความรู้ต่างๆ ที่เรียนรู้ด้วยตนเอง และแนวทางการนำความรู้เหล่านั้นมาใช้ในโครงงาน
