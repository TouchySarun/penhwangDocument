\documentclass[final]{cpecmu}

%% This is a sample document demonstrating how to use the CPECMU
%% project template. If you are having trouble, see "cpecmu.pdf" for
%% documentation.

\projectNo{P046-2}%
\acadyear{2020}

\titleTH{เป็นห่วง (แชทบอทสำหรับการจัดการเวลาทำงานของพนักงาน)}%
\titleEN{Penhwang (Managing employee attendance using LINE chatbot)}

\author{นางสาวธนันพร ยานะ}{Tananporn Yana}{600610739}
\author{นายศรัณญ์ ซือสุวรรณ}{Sarun Suesuwan}{600610777}

\cpeadvisor{navadon}
\cpecommittee{chinawat}
\cpecommittee{dome}

%% Some possible packages to include:
\usepackage[final]{graphicx} % for including graphics

%% Add bookmarks and hyperlinks in the document.
\PassOptionsToPackage{hyphens}{url}
\usepackage[colorlinks=true,allcolors=Blue4,citecolor=red,linktoc=all]{hyperref}

%% Needed just by this example, but maybe not by most reports
\usepackage{afterpage} % for outputting
\usepackage{pdflscape} % for landscape figures and tables. 

%% Some other useful packages. Look these up to find out how to use
%% them.
% \usepackage{natbib}    % for author-year citation styles
% \usepackage{txfonts}
% \usepackage{appendix}  % for appendices on a per-chapter basis
% \usepackage{xtab}      % for tables that go over multiple pages
% \usepackage{subfigure} % for subfigures within a figure
% \usepackage{pstricks,pdftricks} % for access to special PostScript and PDF commands
% \usepackage{nomencl}   % if you have a list of abbreviations

%% if you're having problems with overfull boxes, you may need to increase
%% the tolerance to 9999
% \tolerance=9999

\bibliographystyle{plain}
% \bibliographystyle{IEEEbib}

% \renewcommand{\topfraction}{0.85}
% \renewcommand{\textfraction}{0.1}
% \renewcommand{\floatpagefraction}{0.75}

%% Example for glossary entry
%% Need to use glossary option
%% See glossaries package for complete documentation.
\ifglossary
  \newglossaryentry{lorem ipsum}{
    name=lorem ipsum,
    description={derived from Latin dolorem ipsum, translated as ``pain itself''}
  }
\fi

%% Uncomment this command to preview only specified LaTeX file(s)
%% imported with \include command below.
%% Any other file imported via \include but not specified here will not
%% be previewed.
%% Useful if your report is large, as you might not want to build
%% the entire file when editing a certain part of your report.
% \includeonly{chapters/intro,chapters/background}

\begin{document}
\maketitle
\makesignature

\ifproject
\begin{abstractTH}

เนื่องจากในปัจจุบันบริษัทหลาย ๆ แห่งเริ่มใช้แอปพลิเคชันเพื่อทำการบันทึกเวลาเข้า-ออกงานของพนักงาน แทนการบันทึกโดยใช้กระดาษ, บัตรตอก หรือ เครื่องสแกนนิ้วมือ
\enskip เพื่อความยืดหยุ่นในการเลือกเวลาทำงานของพนักงาน และ ความรวดเร็วในการดึงข้อมูลมาแสดง รวมถึงความสะดวกรวดเร็วในการคำนวณเงินเดือน
\enskip แต่แอปพลิเคชันที่มีอยู่ในตลาดในปัจจุบันก็ยังมีข้อเสียบางประการ เช่น ไม่มีการแจ้งเตือนพนักงานในบางเหตุการณ์ ทำให้แอปดูห่างเหินกับพนักงาน, 
การลงชื่อเข้าหรือออกยังทำได้ยาก มีหลายขั้นตอน ทำให้เวลาเข้า-ออกงานที่บันทึกเข้าสู่ระบบกับเวลาที่พนักงานเข้างานจริงต่างกันพอสมควร 
และ ในบางครั้งอุปกรณ์ของพนักงานอาจไม่มีพื้นที่เพียงพอ หรือ ไม่รองรับการดาวน์โหลดแอปพลิเคชันนั้น ๆ
\enskip เพื่อลดข้อเสียเหล่านี้ทางเราจึงได้พัฒนา เป็นห่วง ซึ่งเป็นไลน์แชทบอทที่ถูกพัฒนาต่อยอดให้สามารถทำงานต่าง ๆ ของแอปพลิเคชันซึ่งกล่าวมาข้างต้นได้
แต่สามารถแก้ข้อเสียที่เกิดขึ้นได้ เช่น มีการแจ้งเตือนมากขึ้น สามารถเข้าถึงได้ง่าย และ ไม่ต้องดาวน์โหลดแอปพลิเคชันเพิ่มเติม
\end{abstractTH}


\begin{abstract}
Nowadays, many companies began to use mobile applications to record time attendance of employees instead of note in paper, punch card or finger scanner 
for increased flexibility in choosing the working hours of employees and speed up access to information including convenience in caluating employee's salary.

\end{abstract}

\iffalse
\begin{dedication}
This document is dedicated to all Chiang Mai University students.

Dedication page is optional.
\end{dedication}
\fi % \iffalse

\begin{acknowledgments}
Your acknowledgments go here. Make sure it sits inside the
\texttt{acknowledgment} environment.

\acksign{2020}{5}{25}
\end{acknowledgments}%
\fi % \ifproject

\contentspage

\ifproject
\figurelistpage

\tablelistpage
\fi % \ifproject

% \abbrlist % this page is optional

% \symlist % this page is optional

% \preface % this section is optional


\pagestyle{empty}\cleardoublepage
\normalspacing \setcounter{page}{1} \pagenumbering{arabic} \pagestyle{cpecmu}

\chapter{\ifcpe บทนำ\else Introduction\fi}

\section{\ifcpe ที่มาของโครงงาน\else Project rationale\fi}
บริษัทที่คิดเงินเดือนจากเวลาทำงานของพนักงาน (นับจากเข้างานจนถึงออกงานในแต่ละวัน) ก็จะมีวิธีการเช็คชื่อเข้า-ออกงานของพนักงานที่แตกต่างกัน
หลากหลายรูปแบบ เช่น การเซ็นชื่อลงบนกระดาษ การใช้บัตรตอก หรือ การสแกนลายนิ้วมือ
แต่ในปัจจุบันมีบริษัทส่วนหนึ่งได้ตระหนักถึงปัญหาจากการใช้วิธีการเช็คชื่อเข้า-ออกงานแบบดังกล่าว
เช่น คำนวณเงินเดือนยากเพราะต้องทำการค้นหาข้อมูลจากเอกสารจำนวนมาก 
พนักงานทุจริตด้วยการตอกบัตรแทนกัน 
หรือ พนักงานตอกบัตรผิดใบ 
ประกอบการใช้มนุษย์ในการบันทึกหรือจัดการข้อมูลมักทำให้เกิดความผิดพลาดที่เกิดจากมนุษย์ (human error) 
ส่งผลให้เกิดความล่าช้า 
จึงมีบริษัทส่วนหนึ่งเลือกที่จะใช้แอปพลิเคชันที่เก็บข้อมูลไว้ในรูปแบบ cloud เพื่อที่จะจัดการแก้ไขปัญหาดังกล่าวข้างต้น 
เพราะ สามารถเรียกดูข้อมูลได้ตลอดเวลา 
เช็คชื่อได้ง่ายขึ้น แต่ 
สามารถป้องกันการทุจริตของพนักงาน 
สรุปผลและคำนวณออกมาเป็นเงินเดือนได้อย่างรวดเร็ว 
รวมไปถึงจัดการการเดินเรื่องขอเอกสารให้มีความสะดวกรวดเร็วมากยิ่งขึ้น และ 
ลดปัญหาความผิดพลาดที่เกิดขึ้นจากการเก็บข้อมูลโดยใช้มนุษย์ไปพร้อมกัน

แต่แอปพลิเคชันที่มีอยู่ในท้องตลาดตอนนี้ก็ยังมีข้อเสีย เช่น 
พนักงานต้องทำการดาวน์โหลดแอปพลิเคชันไว้ในโทรศัพท์ส่วนตัวซึ่งจากผลสำรวจแล้วพนักงานส่วนใหญ่ไม่เต็มใจที่จะดาวน์โหลดแอป ฯ และ 
มีโทรศัพท์ของพนักงานบางคนที่ไม่สามารถโหลดแอป ฯ ได้
ประกอบกับ บางแอปพลิเคชันไม่ได้อำนวยความสะดวกในการใช้งานด้านต่าง ๆ เช่น
ไม่มีการแจ้งเตือนเมื่อพนักงานจะต้องเข้าทำงาน มีการเปลี่ยนแปลงเวลาการทำงานของตนเอง หรือ คำขอต่าง ๆ ของตนเองถูกยืนยัน, ปฎิเสธ
ซึ่งเป็นสิ่งที่แอปพลิเคชันควรจะรองรับ และ 
ปัญหาสำคัญคือ การเช็คชื่อเข้าทำงานของพนักงานยังทำได้ช้ามีหลายขั้นตอนทำให้เวลาที่บันทึกอยู่ในระบบและเวลาที่พนักงานเข้างานจริงต่างกันพอสมควร

ทางผู้พัฒนาเล็งเห็นปัญหาข้างต้นจึงได้พัฒนาโปรเจคนี้ขึ้นโดยการใช้ LINE chatbot มาพัฒนาต่อยอด
เพื่อให้สามารถทำงาน ครอบคลุมฟังก์ชันต่าง ๆ ตามที่แอปพลิเคชันเหล่านั้นทำได้ และควรจะทำได้
โดยรักษาข้อดีต่าง ๆ เอาไว้พร้อมกับแก้ปัญหาที่เกิดขึ้นจากการใช้แอปพลิเคชันเหล่านั้นด้วย

\section{\ifcpe วัตถุประสงค์ของโครงงาน\else Objectives\fi}
\begin{enumerate}
    \item พัฒนาไลน์แชทบอทที่สามารถทำงานเทียบเท่ากับฟังก์ชันหลักของแอปพลิเคชันเช็คชื่อพนักงานที่มีตามท้องตลาดได้ ประกอบด้วย 
    \begin{enumerate}
    \item[1.1] จัดตารางเข้า-ออกงานให้กับพนักงาน
    \item[1.2] เช็คชื่อเข้า-ออกงานโดยการบันทึกสถานที่และเวลา
    \item[1.3] การยื่นคำขอของพนักงาน เช่น เปลี่ยนเวลาการทำงานของตนเอง ขอลา
    \item[1.4] ตั้งค่าบริษัทเช่น การเพิ่ม-ลดพนักงาน กะ แผนก สถานที่ที่จะอนุญาติให้พนักงานเช็คชื่อ และ ประเภทคำขอ
    \end{enumerate} 
    \item มีการแจ้งเตือนเมื่อพนักงานจะต้องเข้าทำงาน, กำลังจะเข้างานสาย, มีการเปลี่ยนแปลงเวลาการทำงานของตนเอง หรือ คำขอต่าง ๆ ของตนเองถูกยืนยัน/ปฎิเสธ
\end{enumerate}

\section{\ifcpe ขอบเขตของโครงงาน\else Project scope\fi}

\subsection{\ifcpe ขอบเขตด้านฮาร์ดแวร์\else Hardware scope\fi}
\begin{itemize}
    \item Android version 4.4 เป็นต้นไป (อุปกรณ์ที่รองรับแอปพลิเคชัน LINE)
\end{itemize}

\subsection{\ifcpe ขอบเขตด้านซอฟต์แวร์\else Software scope\fi}
\begin{itemize}
    \item ระบบบันทึกเวลาการเข้า-ออกงานของพนักงาน
    \item ระบบจัดการคำขอต่าง ๆ ของพนักงาน ประกอบด้วย ขอลา ขอเปลี่ยนกะ และ ขอเข้าร่วมบริษัท
    \item ระบบจัดการตารางเวลาทำงาน 
    \item ระบบจัดการประเภทการลา
    \item ระบบจัดการสถานที่เข้าออกงานของพนักงาน
    \item ระบบจัดการแบ่งกลุ่มพนักงานเป็นแผนก
    \item ระบบแจ้งเตือนเวลาเข้างาน
    \item ระบบแจ้งเตือนเมื่อมีการเปลี่ยนแปลงตารางเวลาการทำงาน
    \item ระบบสรุปประวัติการทำงานของพนักงาน (เพื่อให้ง่ายต่อการคำนวณเงินเดือน)
\end{itemize}

\section{\ifcpe ประโยชน์ที่ได้รับ\else Expected outcomes\fi}
\begin{itemize}
    \item ช่วยอำนวยความสะดวกในการบันทึกเวลาเข้า-ออกงานของพนักงาน
    \item ง่ายต่อการตรวจสอบข้อมูลการเข้า-ออกงาน และ การลางานของพนักงาน
    \item ใช้เวลาในการจัดการคำขอต่าง ๆ น้อยลง
    \item สามารถดูสิทธิ์การลาคงเหลือ และสรุปประวัติการบันทึกเวลาได้
    \item หัวหน้างานสามารถติดตามการเข้า-ออกงานของลูกน้องได้
    \item หัวหน้างานสามารถจัดการหรือเตรียมตัวก่อนล่วงหน้าเมื่อมีพนักงานขอลา
\end{itemize}
\section{\ifcpe เทคโนโลยีและเครื่องมือที่ใช้\else Technology and tools\fi}
\begin{itemize}
    \item Visual Studio code
    \item LINE application
    \item LINE API Messenging
    \item LINE Bot Designer​ \end{englang}
    \item Google dialogflow
    \item Google firebase (firestore)
    \item Nuxt.js
\end{itemize}
\section{\ifcpe แผนการดำเนินงาน\else Project plan\fi}
\begin{plan}{7}{2020}{3}{2021}
    \planitem{7}{2020}{9}{2020}{ศึกษาแอปฯปัจจุบัน}
    \planitem{8}{2020}{9}{2020}{ศึกษาการสร้างแชทบอท​}
    \planitem{7}{2020}{9}{2020}{ออกแบบระบบ}
    \planitem{10}{2020}{10}{2020}{ออกแบบแชทบอท​}
    \planitem{7}{2020}{10}{2020}{ออกแบบเว็บ​}
    \planitem{7}{2020}{10}{2020}{ออกแบบฐานข้อมูล}
    \planitem{11}{2020}{12}{2020}{สร้างแชทบอท}
    \planitem{11}{2020}{1}{2021}{เชื่อมแชทบอทกับฐานข้อมูล}
    \planitem{11}{2020}{1}{2021}{สร้างเว็บ}
    \planitem{11}{2020}{2}{2021}{เชื่อมเว็บกับแชทบอท}
    \planitem{11}{2020}{3}{2021}{ทดสอบและแก้ไข Bugs}
\end{plan}

\section{\ifcpe บทบาทและความรับผิดชอบ\else Roles and responsibilities\fi}
\begin{itemize}
    \item น.ส.ธนันพร ยานะ
    \begin{itemize}
        \item web designer​ \end{englang}
        \item chatbot designer​​ \end{englang}
        \item frontend developer​ \end{englang}
    \end{itemize}
    \item นายศรัณญ์ ซือสุวรรณ
    \begin{itemize} 
        \item database admin
        \item full stack developer​​ \end{englang}
    \end{itemize}
\end{itemize}

\section{\ifcpe%
ผลกระทบด้านสังคม สุขภาพ ความปลอดภัย กฎหมาย และวัฒนธรรม
\else%
Impacts of this project on society, health, safety, legal, and cultural issues
\fi}
การนำเป็นห่วง (แชทบอทสำหรับจัดการเวลาทำงานของพนักงาน) 
มาใช้จะทำให้การจัดการการทำงานพนักงาน ง่าย รวดเร็ว และ เพิ่มประสิทธิภาพในการทำงาน 
ทำให้การเกิดความผิดพลาดที่เกิดจากมนุษย์น้อยลง เนื่องจากใช้ระบบคอมพิวเตอร์ในการจัดการ 
จึงมีความแม่นยำ และ ถูกต้องโดยข้อมูลเหล่านี้จะถูกบันทึกไว้ในระบบคอมพิวเตอร์ทำให้สามารถเรียกใช้ได้ตามต้องการและมีความปลอดภัยเพิ่มขึ้น 
สิ่งเหล่านี้อาจทำให้การปฎิบัติตัวของพนักงานเปลี่ยนไป กล่าวคืออาจตรงต่อเวลามากขึ้น และ 
มีการทำงานแบบเป็นระบบมากขึ้น ตลอดจนทำให้วัฒนธรรมขององค์กรนั้นๆ เปลี่ยนไปในทางที่ดีขึ้นด้วย
\chapter{\ifcpe ทฤษฎีที่เกี่ยวข้อง\else Background Knowledge and Theory\fi}

\quad การทำโครงงาน เริ่มต้นด้วยการศึกษาค้นคว้า ทฤษฎีที่เกี่ยวข้อง หรือ งานวิจัย/โครงงาน ที่เคยมีผู้นําเสนอไว้แล้ว
ซึ่งเนื้อหาในบทนี้ก็จะเกี่ยวกับการอธิบายถึงสิ่งที่เกี่ยวข้องกับโครงงาน 
เพื่อให้ผู้อ่านเข้าใจเนื้อหาในบทถัด ๆ ไปได้ง่ายขึ้น เนื้อหาในบทนี้จะแบ่งออกเป็นสี่ส่วนหลัก ๆ คือ 
แอปพลิเคชันในตลาดปัจจุบัน, LINE และ ส่วนเสริมของ LINE, เทคโนโลยีที่เกี่ยวข้องอื่น ๆ ดังนี้ 
\section{แอปพลิเคชันในตลาดปัจจุบัน}

\subsection{PAYDAY}
\quad SME payday ระบบ HRM (human resources management) สำหรับผู้ประกอบการ SMEs และ พนักงาน 
โดยสามารถจัดการงานเอกสาร และ เข้าถึงงานบุคคลได้สะดวก ทุกที่ทุกเวลา ไม่ว่าจะเป็นการยืนยันเวลาเข้าออกในการทำงาน 
การยื่นคำร้องต่าง ๆ รวมถึงการเบิกค่าใช้จ่ายไปจนถึงการอนุมัติพร้อมทั้งฟังก์ชันใช้งานเฉพาะบุคคลสำหรับการเช็กข้อมูลส่วนตัว 
ทั้งในเรื่องของเงินเดือนรวมถึงภาษี ประกันสังคม เพื่อให้ทุกกระบวนการง่าย สะดวก รวดเร็วยิ่งขึ้น ใช้งานได้หลายรูปแบบทั้งบน website และ mobile application 
\cite{payday}
\begin{itemize}
  \item จุดเด่นของแอปพลิเคชัน
  \begin{itemize}
    \item มีระบบคำนวณและจ่ายเงินเดือนอัตโนมัติ
    \item สามารถสร้างnไฟล์การจ่ายเงินเดือนสำหรับธนาคารไทยพาณิชย์
  \end{itemize}
  \item ความคิดเห็นจากผู้ใช้งานใน App store และ Play store
  \begin{itemize}
    \item ต้องมีการเข้าสู่ระบบก่อนจะเช็คชื่อทุกครั้ง ทำให้ใช้เวลานาน
    \item ระบบเข้าใช้งานไม่เสถียร กรอกข้อมูลเหมือนเดิมทุกอย่างแต่ได้ผลลัพธ์ไม่เหมือนกัน
  \end{itemize}
\end{itemize}

\subsection{JOBCAN}
\quad ระบบจัดการการเข้างานผ่าน cloud system ที่มีบริษัทเลือกใช้งานมากกว่า 10,000 แห่งทั่วโลก 
มีวิธีตอกบัตรเข้างานหลากหลาย เช่น IC card, โทรศัพท์มือถือผ่าน GPS, สแกนลายนิ้วมือ สามารถสร้างผลัดงาน และ 
รับผลัดงานที่พนักงานต้องการผ่านหน้าจอได้โดยตรง รวมถึงตรวจสอบสถานการณ์ทำงานของพนักงานได้แบบ real-time
\cite{jobcan}
\begin{itemize}
  \item จุดเด่นของแอปพลิเคชัน
  \begin{itemize}
    \item มีระบบคำนวณและจ่ายเงินเดือนอัตโนมัติ
    \item สามารถสร้างnไฟล์การจ่ายเงินเดือนสำหรับธนาคารไทยพาณิชย์
  \end{itemize}
  \item ความคิดเห็นจากผู้ใช้งานใน App store และ Play store
  \begin{itemize}
    \item ต้องมีการเข้าสู่ระบบก่อนจะเช็คชื่อทุกครั้ง ทำให้ใช้เวลานาน
    \item ระบบเข้าใช้งานไม่เสถียร กรอกข้อมูลเหมือนเดิมทุกอย่างแต่ได้ผลลัพธ์ไม่เหมือนกัน
  \end{itemize}
\end{itemize}

\subsection{we-la-dee}
\quad เวลาดีทางเลือกใหม่ของระบบบันทึกเวลาทำงานสำหรับ องค์กรที่ทันสมัย 
ใช้สำหรับการบันทึกเวลาเข้าออกงานที่บริษัทของคุณเพื่อรับข้อมูลเวลาการเข้า-ออกงานที่ถูกต้องแบบ Real-time 
พนักงานสามารถบันทึกเวลาเข้า-ออกงานได้ทั้งที่สำนักงานใหญ่ สำนักงานสาขา หรือพื้นที่ทำงานนอกสถานที่ 
ผู้บริหารสามารถตรวจสอบเวลาการเข้า-ออกงานของพนักงาน ชั่วโมงการทำงานได้ทันทีผ่านโทรศัพท์มือถือ 
สามารถดูรายงาน สถิติการทำงาน ได้ทุกที่ทุกเวลา ช่วยให้การบริหารจัดการระบบเวลาการทำงานในองค์กรเป็นเรื่องง่าย 
และพนักงานเองก็สามารถตรวจสอบชั่วโมงการทำงานผ่านโทรศัพท์มือถือของตนเองได้เช่นกัน 
\cite{weladee}
\begin{itemize}
  \item จุดเด่นของแอปพลิเคชัน
  \begin{itemize}
    \item สามารถบันทึกข้อมูลโดยใช้ RFID ได้ 
    \item สามารถแปลงข้อมูลเป็นไฟล์ PDF ได้ 
    \item พนักงานสามารถลงทะเบียนลาพักร้อนได้ โดยระบบจะส่งข้อมูลไปยังแผนก HR ให้อัตโนมัติ
  \end{itemize}
  \item ความคิดเห็นจากผู้ใช้งานใน App store และ Play store
  \begin{itemize}
    \item สะดวกดีครับ ไม่ต้องรอคิวให้เสียเวลา 
    \item แอปพลิเคชันมีการอัปเดตบ่อย 
    \item การสแกนเข้างานยาก บางครั้งใช้เวลานาน  
  \end{itemize}
\end{itemize}

\subsection{OneDee}
\quad วันดี คือ ระบบบริหารจัดการทีมงานผ่าน chatbot ด้วยหน้าจอที่ใช้งานง่าย ช่วยให้การบริหารงาน HR มีประสิทธิภาพมากขึ้น 
พนักงานสามารถลงเวลาทำงาน ตอกบัตร ขาดลา ด้วยการส่งข้อความผ่าน chatbot และ สามารถพูดคุยกันภายในแอพผ่านหน้าจอแชทช่วยให้ทำงานง่ายยิ่งขึ้น 
อีกทั้งยังสามารถให้บริการเวลาทำงานของพนักงานผ่าน AI ลงเวลาตอกบัตร ลงเวลาพนักงาน เพิ่มประสิทธิภาพการบริหารงานด้วย AI ร่วมกับระบบ HR และ chatbot 
ที่สามารถตอบสนอกงการทำงานภายในองค์การ และ ทำงานร่วมกับพนักงานได้อย่างง่ายดาย ทีมงานสามารถพูดคุยสื่อสารภายในองค์กร และขอแบบฟอร์มต่าง ๆ ได้อย่างรวดเร็ว  
\cite{onedee}
\begin{itemize}
  \item จุดเด่นของแอปพลิเคชัน
  \begin{itemize}
    \item ระบบ UI เป็นมิตรสามารถใช้งานได้ง่าย 
    \item สามารถเข้า-ออกงานได้หลายวิธีทั้ง WIFI, GPS, QR, iBeacon 
  \end{itemize}
  \item ความคิดเห็นจากผู้ใช้งานใน App store และ Play store
  \begin{itemize}
    \item หาตำแหน่งเข้า-ออกงานยาก (GPS)  
    \item Log in ยาก ไม่เสถียร  
    \item บางครั้งเกิดปัญหาใช้ QR code เข้า-ออกงานแล้วแอป ฯ catch 
    \item ลงเวลาพร้อมกันหลายคนแล้วมีปัญหา 
  \end{itemize}
\end{itemize}

\subsection{TimeMint}
\quad แอปพลิเคชันบันทึกเข้า-ออกงานทั้งใน และ นอกสถานที่ เพื่อใช้ทดแทนเครื่องตอกบัตร และ เอกสารใบลาทุกประเภท 
เพื่อให้หัวหน้าและพนักงานใช้งาน และ จัดการทั้งเรื่องขาด ลา มาสาย การขออนุมัติปรับปรุงเวลา 
และการบันทึกเวลาเข้า-ออกงานมีการพัฒนาเพื่อให้ครอบคลุมการใช้งานในการจัดการพนักงาน 
เพื่อให้ใช้งานได้อย่างกว้าวงขวาง โดยไม่จำเป็นต้องใช้เครื่องอื่น เพียงการใช้งานผ่านโทรศัพท์มือถือเท่านั้นก็เพียงพอต่อการใช้งานระบบแล้ว  
\cite{timemint}
\begin{itemize}
  \item จุดเด่นของแอปพลิเคชัน
  \begin{itemize}
    \item สามารถเข้า-ออกงานได้หลายวิธีทั้ง WIFI, GPS, QR, iBeacon 
    \item ค่าบริการยืดหยุ่น 
    \item สามารถบันทึกข้อมูลเข้า-ออกงานผ่านทาง LINE ได้ 
  \end{itemize}
  \item ความคิดเห็นจากผู้ใช้งานใน App store และ Play store
  \begin{itemize}
    \item หาตำแหน่งเข้า-ออกงานช้า (GPS) กว่าจะหาเจอก็เข้างานสายแล้ว  
    \item Ux ไม่ friendly กับผู้ใช้ไม่รู้สถานะว่าเข้า-ออกงานอยู่ แอปกินทรัพยากรมาก 
  \end{itemize}
\end{itemize}

\subsection{HumanOs}
\quad เว็บแอปพลิเคชันที่จัดการการเข้า-ออกงานของพนักงานและระบบเงินเดือน 
โดยไม่ต้องใช้เครื่องมือใด ๆ สามารถจัดการวันลาผ่านระบบออนไลน์ 
จัดการคำนวณเงินเดือนรวมถึงระบบภาษี และทุกอย่างยังสามารถ port เป็น excel ได้  
\cite{humanos}
\begin{itemize}
  \item จุดเด่นของแอปพลิเคชัน
  \begin{itemize}
    \item สามารถแปลงข้อมูลเป็นไฟล์ PDF หรือ excel ได้ 
    \item อนุมัติงานต่าง ๆ ผ่านแอปได้ 
  \end{itemize}
  \item ความคิดเห็นจากผู้ใช้งานใน App store และ Play store
  \begin{itemize}
    \item การลาทำได้ยาก 
    \item มีปัญหาการเชื่อมต่อ 
    \item ฟรีกับธุรกิจขนาดเล็ก (ผู้ใช้งานไม่เกิน 10 คน) 
    \item ไม่รองรับบน iOS 
    \item ใช้ง่ายกว่าแสกนนิ้ว 
  \end{itemize}
\end{itemize}

\section{LINE application}
\quad LINE คือ แอปพลิเคชันที่ผสมผสานบริการ Messaging และ Voice Over IP นำมาผนวกเข้าด้วยกัน จึงทำให้เกิดเป็นแอปพลิชันที่สามารถแชท สร้างกลุ่ม ส่งข้อความ โพสต์รูปต่าง ๆ  หรือจะโทรคุยกันแบบเสียงก็ได้  โดยข้อมูลทั้งหมดไม่ต้องเสียเงิน หากเราใช้งานโทรศัพท์ที่มีแพคเกจอินเทอร์เน็ตอยู่แล้ว แถมยังสามารถใช้งานร่วมกันระหว่าง iOS และ Android รวมทั้งระบบปฏิบัติการอื่น ๆ ได้อีกด้วย การทำงานของ LINE นั้น มีลักษณะคล้าย ๆ กับ WhatsApp ที่ต้องใช้เบอร์โทรศัพท์เพื่อยืนยันการใช้งาน แต่ LINE ได้เพิ่มลูกเล่นอื่น ๆ เข้ามา ทำให้ LINE มีจุดเด่นที่เหนือกว่า WhatsApp  

\subsection{จุดเด่นของ LINE} 
\subsubsection{การสนทนาด้วยเสียงฟรี (Free voice calls)} 
\quad ผู้ใช้งานสามารถโทรหาผู้ที่ใช้ LINE ด้วยกันได้ โดยใช้งานผ่านเครือข่าย 3G และ Wi-Fi เพื่อส่งข้อมูลรูปแบบเสียง โดยไม่มีค่าใช้จ่ายใด ๆ  
\subsubsection{ส่งข้อความแบบวิดีโอและเสียง (Send videos \& voice message)} 
\quad นอกจากการแชทด้วยการส่งข้อความแบบปกติแล้ว LINE ยังสามารถอัดภาพวิดีโอหรือเสียงแล้วส่งไปให้เพื่อน ๆ ได้อีกด้วย โดยสามารถส่งได้เป็นคลิปวิดีโอหรือเสียงในแบบสั้น ๆ ความยาวไม่กี่วินาที   
\subsubsection{สติกเกอร์ (Stickers \& Emoticons)}
\quad อีกหนึ่งความสนุกของแชททั่วไปที่ขาดไม่ได้ก็คืออีโมติคอนน่ารัก ๆ ที่ช่วยเพิ่มสีสันให้การแชทสนุกสนานยิ่งขึ้น และสำหรับ LINE มีทั้ง Stickers และ Emoticons รูปแบบต่าง ๆ และยังเลือกดาวน์โหลดเพิ่มเติมได้อีกด้วย ทำให้ผู้ใช้งานหลายคนติดอกติดใจกับ Stickers และ Emoticons น่ารัก ๆ ของ LINE  
\subsubsection{ปรับแต่งภาพพื้นหลัง (Customizable Wallpaper)}
\quad สามารถเปลี่ยน Wallpaper ในหน้าต่างแชทได้ โดยจะมีภาพ Wallpaper มาให้ทั้งหมด 23 แบบ และสามารถเพิ่ม Wallpaper ที่ต้องการ โดยนำรูปที่อยู่ในโทรศัพท์มือถือมาใช้งานเป็น Wallpaper ได้  
\subsubsection{การสนทนาแบบกลุ่ม (Group chat)} 
\quad LINE สามารถสร้างกลุ่มเพื่อพูดคุยกันได้ หากต้องการความเป็นส่วนตัว อยากคุยเฉพาะกลุ่ม LINE ก็สามารถสร้างกลุ่มเอาไว้พูดคุยได้ 
\subsubsection{Timeline} 
\quad LINE มีความเป็นโซเชียลเน็ตเวิร์กในตัว มี Timeline ให้สามารถอัปเดตสถานะ โพสต์รูป คอมเมนต์ หรือกดถูกใจได้เหมือนกับ Facebook เลยทีเดียว 
\subsubsection{การเพิ่มเพื่อน (Add friends / Contacts)}
\quad LINE สามารถเพิ่ม Contacts จากรายชื่อในโทรศัพท์หากมีเพื่อนคนไหนใช้แอปพลิเคชันนี้อยู่ จะมีสัญลักษณ์ LINE แสดงให้เห็นและสามารถเพิ่มเป็นเพื่อนได้ทันที QR Code สามารถสแกน QR Code ของเพื่อนเพื่อเพิ่มเป็นเพื่อนใน LINE และสามารถสร้าง QR Code ของเราเอง เพื่อใช้สำหรับให้เพื่อน ๆ คนอื่น มาสแกน QR Code เพื่อเพิ่มเพื่อนใน LINE ได้ Shake it! เขย่าโทรศัพท์มือถือ เป็นวิธีการเพิ่มเพื่อนที่เจ๋งสุด ๆ ของ LINE ใช้ในกรณีที่ทั้งสองโทรศัพท์สองเครื่องอยู่ด้วยกัน เมื่อเขย่าเครื่องพร้อม ๆ กัน ก็สามารถเพิ่มเป็นเพื่อนกันได้ Search by ID คือ เราสามารถค้นหาเพื่อนได้จาก ID (คล้าย ๆ กับ PIN ของ BB) โดยการพิมพ์ ID ของเพื่อนที่ต้องการ 

\subsection{LINE Messaging API} 
\quad Line Messaging API คือ การสื่อสารระหว่างบริการของคุณและผู้ใช้ LINE เป็นการสื่อสารแบบสองฝ่าย จะทำให้คุณสามารถให้บริการได้ในห้องแชท LINE เพื่อการให้บริการที่เหมาะสมสำหรับผู้ใช้ LINE แต่ละคนและ Messaging API จะส่งและรับข้อมูลระหว่าง Server ของคุณและแอพ LINE ผ่านทาง Server ของทาง LINE การส่งคำขอจะใช้ API แบบ JSON Messaging API ทำการเชื่อมต่อระหว่าง User ผ่านทาง LINE official account ซึ่ง Messaging API จะสามารถตอบรับเพื่อนรวมถึงส่งข้อความหา User คนอื่น ๆ ที่ Add account เราเป็นเพื่อนโดยผ่านหน้า LINE Manager ที่เราตั้งไว้หรือส่งออกจากจาก Server ของเราก็ได้ในรูปแบบ interactive โต้ตอบ  การใช้งาน Messaging API ทำให้คุณสามารถส่งข้อมูลระหว่าง Server ของเรา ไปยัง User LINE ผ่านทาง LINE Platform ซึ่ง Request ที่ใช้ส่งข้อมูลต้องอยู่ในรูป JSON format โดยตัว Server เราจะต้องเชื่อมต่อกับ LINE Platform และเมื่อ มี User เพิ่ม Account LINE เราเป็นเพื่อน หรือ ส่งข่อความมาหาเรา ทาง LINE Platform จะทำการส่ง Request มายัง Server ที่เราลงทะเบียนผูกไว้กับ LINE account นั้นทันที วิธีนี้เรียกว่า Webhook ซึ่งมันทำให้ผู้ใช้งานรู้สึกเหมือนกับว่าได้โต้ตอบกับคนจริง ๆ 
\cite{lineAPI}
\subsection{LINE Bot} 
\quad LINE Bot คือ Line Official Account ที่ได้นำ Messaging API มาใช้ เป็นบริการ API ตัวหนึ่งที่เปิดให้บริการสำหรับนักพัฒนา โดยเจ้าของ Line Official Account จะทำการกำหนดหรือตั้งค่าไว้ด้านหลังบ้านของบริการ เพื่อให้สามารถโต้ตอบกับผู้ใช้งานได้โดยที่ไม่ต้องใช้คนมาเป็นคนตอบ ซึ่งนี่คือข้อดีของการใช้บริการตอนนี้ เพราะนอกจากจะทำให้ผู้ใช้ใช้งานได้ง่ายมากขึ้นแล้ว ผู้ที่เป็นแอดมินก็จะสะดวกสบายมากขึ้นเช่นกัน เพราะไม่ต้องมาคอยตอบคำถามที่ถามซ้ำ ๆ หรือไม่จำเป็นต้องมานั่งเก็บข้อมูลทีละคน ช่วยให้ผู้ใช้งานแก้ไขปัญหาได้ในเบื้องต้นอย่างว่องไว ไม่ต้องรอคอยเป็นเวลานาน สร้างความประทับใจ ปิดการขายได้เร็วขึ้นและลดต้นทุนในการจ้างแอดมินเพื่อมาคอยตอบคำถามตลอดเวลา เพราะบริการนี้จะช่วยเหลือคุณได้ทุกอย่างที่สามารถทำได้ 
\subsubsection{การสร้าง LINE Bot โดยใช้ Dialogflow} 
\quad ประกอบด้วย 2 ส่วน คือ LINE Official Account เป็นส่วนที่เราต้องสร้างขึ้นเพื่อใช้ในการทำ LINE Bot ที่ไว้ใช้ในการโต้ตอบกับ Dialogflow สามารถกำหนดได้ทั้ง สติกเกอร์ รูปภาพ ข้อความ และ วีดีโอ และ Dialogflow เป็นแพลตฟอร์มที่สามารถช่วยในการพัฒนา LINE Bot ได้สามารถแบ่งได้ 2 กรณีดังนี้ 1.การเขียน Dialogflow ขั้นพื้นฐานไม่จำเป็นต้องทำการเขียนโปรแกรมเลย เนื่องจากเราสามารถพิมพ์ข้อความต่าง ๆที่ใช้สำหรับการถาม-ตอบได้เลย 2.การเขียน Dialogflow ขั้นสูงอาจจะมีการเขียนโปรแกรมเพื่อเพิ่มความสามารถของ LINE Bot ได้ เช่น การส่ง Location การส่งรูปภาพ การส่งสติกเกอร์ เป็นต้น 
\cite{lineBot} \cite{lineDf} \cite{lineDfPyFb}
\section{เทคโนโลยีที่เกี่ยวข้องอื่น ๆ}

\subsubsection{Visual Studio Code (VS Code)}
\quad Visual Studio Code หรือ VS Code เป็นโปรแกรม Code Editor ที่ใช้ในการแก้ไขและปรับแต่งโค้ด โดยมาจากค่ายไมโครซอฟท์ ที่มีการพัฒนาออกมาในรูปแบบของ Opensource จึงสามารถนำมาใช้งานได้แบบฟรี ๆ ทีสต้องการความเป็นมืออาชีพ ซึ่ง Visual Studio Code นั้น เหมาะสำหรับนักพัฒนาโปรแกรมที่ต้องการใช้งานกับแพลตฟอร์ม มีการรองรับการใช้งานทั้งบน Windows, macOS และ Linux มีการสนับสนุนทั้งภาษา  JavaScript, TypeScript และ Node.js สามารถเชื่อมต่อกับ  Git ได้ สามารถนำมาใช้งานได้ง่ายไม่ซับซ้อน มีเครื่องมือส่วนขยายต่าง ๆ ให้เราเลือกใช้อย่างมาก ไม่ว่าจะเป็น 1.การเปิดใช้งานภาษาอื่น ๆ ทั้ง ภาษา  C++,  C\#, Java, Python, PHP หรือ Go  2.Themes 3.Debugger 4.Commands 
\cite{vscode}
\subsubsection{Google Dialogflow} 
\quad Dialogflow คือ Platform สำหรับสร้าง chatbot ของ Google ที่ใช้ machine learning ด้าน Natural Language Processing (NLP) มาช่วยในทำความเข้าใจถึงความต้องการ (Intent) และสิ่งที่ต้องการ (Entity) ในประโยคสนทนาของผู้ใช้งานและตอบคำถามตามความต้องการของผู้ใช้งาน ตามกฎที่ผู้พัฒนาวางเอาไว้ ซึ่ง Dialogflow จะช่วยเพิ่มความยืดหยุ่นของประโยคที่ Chatbot รับมา ว่าไม่จำเป็นต้องตรงตามเงื่อนไขก็สามารถเข้าใจถึงความต้องการของผู้ใช้งานได้ 
\cite{lineDf}
\subsubsection{Google Firebase} 
\quad Firebase คือ Platform ที่รวบรวมเครื่องมือต่าง ๆ สำหรับการจัดการในส่วนของ Backend (Server side) ซึ่งทำให้สามารถ Build Mobile Application ได้อย่างมีประสิทธิภาพ และยังลดเวลาและค่าใช้จ่ายของการทำ Server side หรือการวิเคราะห์ข้อมูลให้อีกด้วย โดยมีทั้งเครื่องมือที่ฟรี และเครื่องมีที่มีค่าใช้จ่าย (สำหรับการต่อขยาย) 
\cite{firebase}
\subsubsection{Nuxt.js }
\quad Nuxt.js คือ Framework ที่นำ Vue.js มาสร้าง web application เสริมความสามารถในการทำ SSR และ Progressive Web Application (PWA) 
\cite{nuxt}

% \section{Third section}
% Section 3 text. The dielectric constant\index{dielectric constant}
% at the air-metal interface determines
% the resonance shift\index{resonance shift} as absorption or capture occurs
% is shown in Equation~\eqref{eq:dielectric}:

% \begin{equation}\label{eq:dielectric}
% k_1=\frac{\omega}{c({1/\varepsilon_m + 1/\varepsilon_i})^{1/2}}=k_2=\frac{\omega
% \sin(\theta)\varepsilon_\mathit{air}^{1/2}}{c}
% \end{equation}

% \noindent
% where $\omega$ is the frequency of the plasmon, $c$ is the speed of
% light, $\varepsilon_m$ is the dielectric constant of the metal,
% $\varepsilon_i$ is the dielectric constant of neighboring insulator,
% and $\varepsilon_\mathit{air}$ is the dielectric constant of air.

% \section{About using figures in your report}

% % define a command that produces some filler text, the lorem ipsum.
% \newcommand{\loremipsum}{
%   \textit{Lorem ipsum dolor sit amet, consectetur adipisicing elit, sed do
%   eiusmod tempor incididunt ut labore et dolore magna aliqua. Ut enim ad
%   minim veniam, quis nostrud exercitation ullamco laboris nisi ut
%   aliquip ex ea commodo consequat. Duis aute irure dolor in
%   reprehenderit in voluptate velit esse cillum dolore eu fugiat nulla
%   pariatur. Excepteur sint occaecat cupidatat non proident, sunt in
%   culpa qui officia deserunt mollit anim id est laborum.}\par}

% \begin{figure}
%   \centering

%   \fbox{
%      \parbox{.6\textwidth}{\loremipsum}
%   }

%   % To include an image in the figure, say myimage.pdf, you could use
%   % the following code. Look up the documentation for the package
%   % graphicx for more information.
%   % \includegraphics[width=\textwidth]{myimage}

%   \caption[Sample figure]{This figure is a sample containing \gls{lorem ipsum},
%   showing you how you can include figures and glossary in your report.
%   You can specify a shorter caption that will appear in the List of Figures.}
%   \label{fig:sample-figure}
% \end{figure}

% \loremipsum\loremipsum

% % This code demonstrates how to get a landscape table or figure. It
% % uses the package lscape to turn everything but the page number into
% % landscape orientation. Everything should be included within an
% % \afterpage{ .... } to avoid causing a page break too early.
% \afterpage{
%   \begin{landscape}
%   \begin{table}
%     \caption{Sample landscape table}
%     \label{tab:sample-table}

%     \centering

%     \begin{tabular}{c||c|c}
%         Year & A & B \\
%         \hline\hline
%         1989 & 12 & 23 \\
%         1990 & 4 & 9 \\
%         1991 & 3 & 6 \\
%     \end{tabular}
%   \end{table}
%   \end{landscape}
% }

% \loremipsum\loremipsum\loremipsum

% \section{Overfull hbox}

% When the \verb.semifinal. option is passed to the \verb.cpecmu. document class,
% any line that is longer than the line width, i.e., an overfull hbox, will be
% highlighted with a black solid rule:
% \begin{center}
% \begin{minipage}{2em}
% juxtaposition
% \end{minipage}
% \end{center}

% \section{\ifcpe%
% ความรู้ตามหลักสูตรซึ่งถูกนำมาใช้หรือบูรณาการในโครงงาน
% \else%
% ISNE knowledge used, applied, or integrated in this project
% \fi
% }

% อธิบายถึงความรู้ และแนวทางการนำความรู้ต่างๆ ที่ได้เรียนตามหลักสูตร ซึ่งถูกนำมาใช้ในโครงงาน

% \section{\ifcpe%
% ความรู้นอกหลักสูตรซึ่งถูกนำมาใช้หรือบูรณาการในโครงงาน
% \else%
% Extracurricular knowledge used, applied, or integrated in this project
% \fi
% }

% อธิบายถึงความรู้ต่างๆ ที่เรียนรู้ด้วยตนเอง และแนวทางการนำความรู้เหล่านั้นมาใช้ในโครงงาน

\chapter{\ifproject%
\ifcpe โครงสร้างและขั้นตอนการทำงาน\else Project Structure and Methodology\fi
\else%
\ifcpe โครงสร้างและขั้นตอนการทำงาน\else Project Structure\fi
\fi
}

% ในบทนี้จะกล่าวถึงหลักการ และการออกแบบระบบ

\makeatletter

% \renewcommand\section{\@startsection {section}{1}{\z@}%
%                                    {13.5ex \@plus -1ex \@minus -.2ex}%
%                                    {2.3ex \@plus.2ex}%
%                                    {\normalfont\large\bfseries}}

\makeatother
%\vspace{2ex}
% \titleformat{\section}{\normalfont\bfseries}{\thesection}{1em}{}
% \titlespacing*{\section}{0pt}{10ex}{0pt}

\section{แผนการทำงาน}
\quad ใช้รูปแบบ software development models เป็นรูปแบบ scrum โดยแบ่งการทำงานเป็นรอบ 
โดยที่แต่ละรอบการทำงานมีระยะเวลา 1 สัปดาห์ในช่วงท้ายของแต่ละสัปดาห์จะมีการตรวจสอบความคืบหน้าของการดำเนินงานในส่วนต่าง ๆ 
และ ทุก ๆ ท้ายของรอบการทำงานจะมีการประชุมทบทวนแผนงานที่จะทำสำหรับรอบถัดไปร่วมกันเพื่อยืนยันแผนการดำเนินงานหากมีการเปลี่ยนแปลงเกิดขึ้น 
รวมถึงสรุปข้อผิดพลาดจากการทำงานในรอบนี้ด้วย
\section{การทำงานของระบบ}
\begin{figure}
\begin{center}
\includegraphics[width=10cm,keepaspectratio]{./images/penhwang_er_diagram_2.jpg}
\end{center}
\caption[แสดง use case diagram]{แสดง use case diagram ซึ่งแสดงให้เห็นกลุ่มผู้ใช้งานทั้ง 2 กลุ่มประกอบด้วย พนักงานทั่วไป (employee) และ พนักงาน hr (human resources employee)}
\end{figure}

\begin{figure}
\begin{center}
\includegraphics[width=14cm,keepaspectratio]{./images/penhwang_db_schema.jpg}
\end{center}
\caption[แสดง E-R diagram]{แสดง E-R diagram (Entity-Relationship diagram) ซึ่งแสดงรายละเอียดของข้อมูลของแต่ละ entity ประกอบด้วย พนักงาน และ บริษัท}
\end{figure}

\begin{figure}
\begin{center}
\includegraphics[width=14cm,keepaspectratio]{./images/structure.jpg}
\end{center}
\caption[แสดงโครงสร้างของระบบ]{
  แสดงโครงสร้างของระบบ ซึ่งแบ่งเป็น 2 ส่วนใหญ่ ๆ คือ 
  front-end ประกอบด้วย LINE application (LINE bot) และ Nuxt.js 
  และ back-end (Google firebase) ซึ่งประกอบด้วย firebase cloud functions และ firebase firestore 
  โดย มี Google dialogflow ขั้นกลางระหว่าง LINE application และ firebase เพื่อทำหน้าที่แปลความหมายข้อความ
}
\end{figure}
\chapter{\ifproject%
\ifcpe การทดลองและผลลัพธ์\else Experimentation and Results\fi
\else%
\ifcpe การประเมินระบบ\else System Evaluation\fi
\fi}

ระบบที่ถูกพัฒนาขึ้นมาจะต้องสามารถทำได้ดังนี้ 

\section{การบันทึกเวลาเข้า-ออกของพนักงาน} 
ข้อมูลที่ถูกบันทึกต้องถูกต้องครบถ้วน เพื่อที่จะสามารถนำไปใช้ในการจัดการด้านต่าง ๆ และคำนวณเงินเดือนได้อย่างมีประสิทธิภาพ 

\section{การจัดการตารางเวลาทำงาน}
ข้อมูลที่ถูกบันทึกในตารางจะต้องถูกต้อง มีประสิทธิภาพ และเมื่อมีการเปลี่ยนแปลงตารางเวลาทำงานจะต้องมีการแจ้งเตือนไปยังเจ้าของตารางเพื่อให้รับรู้เข้าใจตรงกัน 

\section{ด้านความแม่นยำในการคำนวณเงินเดือน} 
ระบบต้องสามารถคำนวณได้ว่าเงินเดือนของพนักงานแต่ละคนเป็นเท่าใด โดยพิจารณาจากจำนวนชั่วโมงการทำงานของพนักงาน 

\chapter{\ifcpe บทสรุปและข้อเสนอแนะ\else Conclusions and Discussion\fi}

\section{\ifcpe สรุปผล\else Conclusions\fi}
\quad “เป็นห่วง” เป็นระบบที่พัฒนาขึ้นเพื่อเป็นทางเลือกในการจัดการเวลาทํางานของพนักงานสําหรับบริษัทที่ต้องการใช้เทคโนโลยีใหม่ ๆ โดยพนักงานสามารถเข้าถึงได้ง่าย ผ่านไลน์แชทบอท 
ดังนั้นจึงไม่ต้องติดตั้งแอปพลิเคชันเพิ่มเติม ระบบออกแบบให้ช่วยลดความผิดพลาดของมนุษย์ จึงง่ายต่อการจัดการ และ ตรวจสอบ ความถูกต้องโดยฝ่ายบุคคล
โดย เป็นห่วง ได้มีการทดสอบการใช้งานเบื้องต้นแล้ว พบว่ามีประโยชน์ต่อผู้ใช้ และ ง่ายต่อการใช้งาน ทั้งนี้ในอนาคตจะต้องมีการทดสอบ 
และสํารวจผลเพื่อตรวจสอบความถูกต้อง และ ความแม่นยําของระบบเพิ่มเติม เนื่องจากสถานการณ์ COVID-19 ในปัจจุบันทําให้ไม่สามารถทดสอบกับผู้ใช้งานทุกบทบาทได้

\section{\ifcpe ปัญหาที่พบและแนวทางการแก้ไข\else Challenges\fi}

% ในการทำโครงงาน เป็นห่วง ไลน์แชทบอทสำหรับจัดการเวลาทำงานของพนักงาน พบว่าเกิดปัญหาหลัก ๆ ดังนี้
% \begin{itemize}
%   \item ข้อมูลมีความสัมพันธ์กันซับซ้อนไม่เหมาะกับการเก็บข้อมูลแบบ NoSQL
%   \item การเชื่อมต่อกันหลายขั้นหลายระบบทำให้การตอบสนองกับผู้ใช้งานช้าลง
%   \item LINE ยังไม่มีรูปแบบที่รองรับสำหรับการทำงานที่ออกแบบไว้ทั้งหมด
%   \item การพัฒนาระบบได้ช้า เนื่องจากไม่ได้ศึกษาเทคโนโลยีที่จะใช้แบบลึกซึ้งก่อนทำงานจริง
% \end{itemize}
ในการทำโครงงาน เป็นห่วง ไลน์แชทบอทสำหรับจัดการเวลาทำงานของพนักงาน พบว่าเกิดปัญหาที่สำคัญประกอบด้วย 
ข้อมูลของพนักงาน บริษัท และ คำขอต่าง ๆ มีความสัมพันธ์กันซับซ้อนไม่เหมาะกับการเก็บข้อมูลแบบ NoSQL 
แต่สาเหตุที่ผู้พัฒนาเลือกใช้ฐานข้อมูลแบบ NoSQL เพราะในช่วงการออกแบบไม่ได้คำนึกถึงการใช้งานทั้งหมด และ firebase 
มีข้อดีคือสามารถ deploy ฐานข้อมูล และ API ที่เขียนขึ้นมาได้ทันที และ ฐานข้อมูลแบบ NoSQL มีความยืดหยุ่นสามารถรองรับความผิดพลาดจากการใส่ข้อมูลได้ 
ปัญหาต่อมาคือ การเชื่อมต่อกันหลายขั้นหลายระบบทำให้การตอบสนองกับผู้ใช้งานช้าลง 
อีกปัญหาคือ แอปพลิเคชัน LINE ยังไม่มีรูปแบบที่รองรับสำหรับการทำงานที่ออกแบบไว้ทั้งหมด 
ปัญหาที่สำคัญสุดท้ายคือ การพัฒนาระบบได้ช้า เนื่องจากไม่ได้ศึกษาเทคโนโลยีที่จะใช้แบบลึกซึ้งก่อนทำงานจริง

\section{\ifcpe%
ข้อเสนอแนะและแนวทางการพัฒนาต่อ
\else%
Suggestions and further improvements
\fi
}

% ข้อเสนอแนะเพื่อพัฒนาโครงงานนี้ต่อไป หรือ สำหรับผู้พัฒนาทีมอื่น ๆ ที่จะทำโครงงานที่คล้ายกันในอนาคต มีดังนี้
% \begin{itemize}
%   \item ในขั้นตอนการออกแบบระบบ ให้คำนึงถึงการนำข้อมูลไปใช้งานด้วย โดยหากการดึงข้อมูลแต่ละครั้งมีความจำเป็นต้องใช้ข้อมูลจากหลายแหล่งพร้อม ๆ กันควรเป็นการเก็บแบบ SQL มากกว่าเพื่อความสะดวกในการจัดการข้อมูล
%   \item หากเลือกที่จะใช้แบบ NoSQL ก็ควรศึกษาประโยชน์ และ ข้อเสีย รวมถึงข้อควรระวังในการออกแบบฐานข้อมูลแบบนี้ด้วย
%   \item นอกจากศึกษาเทคโนโลยีที่จะใช้แล้ว ควรมีการลองนำเทคโนโลยีเหล่านั้นมาใช้งานกับโปรเจคเล็ก ๆ ก่อน เพื่อจะเปลี่ยนแปลง แก้ไขได้ทันที
%   \item หากจะให้ "การใช้งานง่าย" เป็นจุดเด่นของโปรเจค ควรมีการทำ prototype ไปทดสอบกับผู้ใช้งานบ่อย ๆ เพื่อนำข้อคิดเห็นมาปรับปรุง
% \end{itemize}


ข้อเสนอแนะเพื่อพัฒนาโครงงานนี้ต่อไป หรือ สำหรับผู้พัฒนาทีมอื่น ๆ ที่จะทำโครงงานที่คล้ายกันในอนาคต ประกอบด้วย 
ในขั้นตอนการออกแบบระบบ ผู้พัฒนาควรคำนึงถึงการนำข้อมูลไปใช้งานทั้งหมด หมายความว่าต้องออกแบบระบบ และ ฟังก์ชันทั้งหมดก่อนจะเลือกใช้ฐานข้อมูล 
โดยหากการดึงข้อมูลไปใช้งานแต่ละครั้งมีความจำเป็นต้องใช้ข้อมูลจากหลายจุดพร้อม ๆ กันควรเป็นการเก็บแบบ SQL มากกว่าเพื่อความสะดวกในการจัดการข้อมูล 
แต่หากเลือกที่จะใช้แบบ NoSQL ก็ควรศึกษาประโยชน์ และ ข้อเสีย รวมถึงข้อควรระวังในการออกแบบฐานข้อมูลแบบนี้ด้วย 
ต่อไปคือ นอกจากศึกษาเทคโนโลยีที่จะใช้แล้ว ควรมีการลองนำเทคโนโลยีเหล่านั้นมาใช้งานกับโครงงานเล็ก ๆ ก่อน เพื่อให้รู้ผลจากการเพิ่มเทคโนโลยีนี้เข้าไปในระบบ และ เพื่อให้รู้วิธีการใช้งานที่ถูกต้อง 
คำแนะนำข้อสุดท้ายคือ หากผู้พัฒนาต้องการให้ "การใช้งานง่าย" เป็นจุดเด่นของโครงการควรมีการทำ prototype ไปทดสอบกับผู้ใช้งานบ่อย ๆ เพื่อนำข้อคิดเห็นมาปรับปรุง จะส่งผลให้ผลสุดท้ายของโครงการออกมาตรงตามความต้องการของผู้ใช้งานมากกว่าตอนนี้


\bibliography{sampleReport}

\ifproject
\appendix
\chapter{แผนภาพการออกแบบระบบ}

ในขั้นตอนทำโครงงาน เป็นห่วง เริ่มจากการออกแบบก่อนนำไปสร้างเป็นระบบจริง เนื้อหาในภาคผนวกนี้ประกอบด้วยขั้นตอนการออกแบบการทำงานของฟังก์ชันต่าง ๆ รวมถึงหน้าตาของระบบ ที่เกี่ยวข้องกับโครงงาน 
โดยเนื้อหาจะแบ่งออกเป็นสองส่วนหลัก ๆ คือ 
การออกแบบวิธีการทำงานของระบบซึ่งออกแบบโดยใช้แอปพลิเคชัน Goodnotes และการออกแบบหน้าตาของระบบซึ่งออกแบบโดยใช้โปรแกรม Adobe XD ดังนี้
\begin{figure}
  \begin{center}
    \includegraphics[width=\linewidth]{./images/design1.jpg}
  \end{center}
  \caption[รูปแสดงภาพร่างของ chatbot]{รูปแสดงภาพร่างของ chatbot} 
\end{figure}

\begin{figure}
  \begin{center}
    \includegraphics[width=14cm,keepaspectratio]{./images/design3.jpg}
  \end{center}
  \caption[รูปแสดงภาพร่างของเว็บแอปพลิเคชัน]{รูปแสดงภาพร่างของเว็บแอปพลิเคชัน} 
  
\end{figure}

\begin{figure}
  \begin{center}
    \includegraphics[width=14cm,keepaspectratio]{./images/design4.jpg}
  \end{center}
  \caption[รูปแสดงภาพร่างของเว็บแอปพลิเคชัน2]{รูปแสดงภาพร่างของเว็บแอปพลิเคชัน2} 
  
\end{figure}

\begin{figure}
  \begin{center}
    \includegraphics[width=14cm,keepaspectratio]{./images/design5.jpg}
  \end{center}
  \caption[รูปแสดงภาพร่างของเว็บแอปพลิเคชัน3]{รูปแสดงภาพร่างของเว็บแอปพลิเคชัน3} 
  
\end{figure}

\begin{figure}
  \begin{center}
    \includegraphics[width=14cm,keepaspectratio]{./images/design7.jpg}
  \end{center}
  \caption[รูปแสดงลำดับขั้นตอนการทำงานและการตรวจสอบความเรียบร้อย]{รูปแสดงลำดับขั้นตอนการทำงานและการตรวจสอบความเรียบร้อย} 
  
\end{figure}

\begin{figure}
  \begin{center}
    \includegraphics[width=\linewidth]{./images/design_xd.jpg}
  \end{center}
  \caption[รูปแสดงผลจากการออกแบบด้วยโปรแกรม Adobe XD]{รูปแสดงผลจากการออกแบบด้วยโปรแกรม Adobe XD} 
  
\end{figure}

\begin{figure}
  \begin{center}
    \includegraphics[width=\linewidth]{./images/design_xd2.jpg}
  \end{center}
  \caption[รูปแสดงผลจากการออกแบบด้วยโปรแกรม Adobe XD 2]{รูปแสดงผลจากการออกแบบด้วยโปรแกรม Adobe XD 2} 
  
\end{figure}
% \chapter{\ifcpe คู่มือการใช้งานระบบ\else Manual\fi}
% ฟังก์ชันที่โปรเจคนี้รองรับ และ วิธีการใช้งานมีดังนี้
% \section{สร้างบริษัทของคุณ}



% \section{code จาก GitHub}
% โปรเจคนี้ถูกแบ่งออกเป็น 3 repositories คือ 

% penhwang backend

% \url{https://github.com/TouchySarun/penhwang_backend}, 

% penhwang frontend mobile

% \url{https://github.com/Tiewly/penhwang_frontend_mobile} 

% และ penhwang frontend desktop 

% \url{https://github.com/TouchySarun/penhwang_frontend_desktop} 
% \section{โปรแกรม และ ส่วนเสริมที่จำเป็น}
% \subsection{IDE (Integrated development environment)}
% โดยโปรเจคนี้เลือกใช้ vs code เป็น IDE หาวิธี install ได้จาก \url{https://code.visualstudio.com/}
% \subsection{package manger command}
% โดยโปรเจคนี้เลือกใช้ npm เป็น package manager หาวิธี install ได้จาก npm 

% \url{https://nodejs.org/en/}
% \subsection{ส่วนเสริม}
% หลังจาก install vs code และ npm แล้ว ให้เปิด vs code ขึ้นมาและเปิด terminal โดยใช้คำสั่งลัด "ctrl + shift + p" 
% \begin{itemize}
%   \item nuxt พิมพ์คำสั่ง > npm i -g nuxt
%   \item firebase-npm เปิดไฟล์ penhwang\_backend โดย คลิกที่ File/Open Floder ... แล้วเลือกไฟล์
% จากนั้นเปิด terminal แล้ว install ส่วนเสริม โดยใช้คำสั่ง > npm i -g firebase
%   \item vue2-google-maps เปิดไฟล์ penhwang\_frontend\_desktop โดย คลิกที่ File/Open Floder ... แล้วเลือกไฟล์
% จากนั้นเปิด terminal แล้ว install ส่วนเสริม โดยใช้คำสั่ง 

% > npm i vue2-google-maps
% \end{itemize}
% \subsection{key สำหรับ API ต่าง ๆ}
% ตัวโปรเจคนี้มีความจำเป็นที่จะต้องเชื่อมต่อกับหลาย ๆ API(Application Programming Interface) เช่น firebase และ google map 
% และ แต่ละ API ต้องมี key ในการเชื่อมต่อ โดยต้องนำ key มาจากที่ต่าง ๆ ดังนี้
% \begin{itemize}
%   \item firebase โดยไปที่หน้า firebase console \url{https://console.firebase.google.com/} หากยังไม่มีโปรเจคของตนเองให้ทำการเพิ่มโปรเจคใหม่จากนั้นเข้าไปที่หน้า Project settings/general สิ่งที่ต้องการมีดังนี้
%   \begin{itemize}
%     \item apiKey
%     \item authDomain
%     \item databaseURL
%     \item projectId
%     \item storageBucket
%     \item messagingSenderId
%     \item appendixmeasurementId
%   \end{itemize}
%   นำทุกอย่างที่ต้องการใส่ในไฟล์ db.js
% \end{itemize}
% \begin{itemize}
%   \item firebase โดยไปที่หน้า firebase console \url{https://console.firebase.google.com/} หากยังไม่มีโปรเจคของตนเองให้ทำการเพิ่มโปรเจคใหม่จากนั้นเข้าไปที่หน้า Project settings/general สิ่งที่ต้องการมีดังนี้
%   \begin{itemize}
%     \item apiKey
%     \item authDomain
%     \item databaseURL
%     \item projectId
%     \item storageBucket
%     \item messagingSenderId
%     \item appendixmeasurementId
%   \end{itemize}
%   นำทุกอย่างที่ต้องการใส่ในไฟล์ db.js
% \end{itemize}



%% Display glossary (optional) -- need glossary option.
\ifglossary\glossarypage\fi

%% Display index (optional) -- need idx option.
\ifindex\indexpage\fi

\begin{biosketch}
  \begin{center}
    \includegraphics[width=1.5in]{images/tananporn.jpg}
  \end{center}
  
  นางสาวธนันพร ยานะ เกิดเมื่อวันที่ 21 ธันวาคม 2541 ณ จังหวัดเชียงใหม่ สำเร็จ
  การศึกษาระดับมัธยมจากโรงเรียนยุพราชวิทยาลัย เข้าศึกษาที่ภาควิชาวิศวกรรมคอมพิวเตอร์
  มหาวิทยาลัยเชียงใหม่ เมื่อ สิงหาคม 2559 โดยมีความสนใจเป็นพิเศษในด้านการเขียน และออกแบบเว็บแอปพลิเคชัน
  ระหว่างศึกษาได้เข้าร่วมกิจกรรมต่าง ๆ ทั้งด้านวิชาการและกีฬา เข้าร่วมกิจกรรมต่าง ๆ ที่คณะจัด
  ขึ้นอยู่เสมอ เป็นสมาชิกชมรมเทนนิส และชมรมคอมพิวเตอร์ คณะวิศวกรรมศาสตร์
\end{biosketch}

\begin{biosketch}
  \begin{center}
    \includegraphics[width=1.5in]{images/sarun.jpg}
  \end{center}
  
  นายศรัณญ์ ซือสุวรรณ เกิดเมื่อวันที่ 18 กุมภาพันธ์ 2542 ณ จังหวัดพะเยา สำเร็จ
  การศึกษาระดับมัธยมจากโรงเรียนพะเยาพิทยาคม เข้าศึกษาที่ภาควิชาวิศวกรรมคอมพิวเตอร์
  มหาวิทยาลัยเชียงใหม่ เมื่อ สิงหาคม 2559 โดยมีความสนใจเป็นพิเศษในด้านการเขียนเว็บแอปพลิเคชัน ทั้งทาง front-end และ back-end
  ระหว่างศึกษาได้เข้าร่วมกิจกรรมต่าง ๆ ทั้งด้านวิชาการและกีฬา เข้าร่วมกิจกรรมต่าง ๆ ที่คณะจัด
  ขึ้นอยู่เสมอ เป็นสมาชิกชมรมชมรมคอมพิวเตอร์ คณะวิศวกรรมศาสตร์
\end{biosketch}
\fi % \ifproject
\end{document}
